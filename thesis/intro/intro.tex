\chapter{Introduction (draft)}
\label{ch:intro}

% TODO: also have a look at the masters talk write up from 2015

\section{Obesity}
\label{sec:obesity}

% General intro to obesity.
% Implications of obesity -- disease associated with it and/or cancer.
Obesity has been one of the major global problems for more than a decade, associated with many noncommunicable diseases such as diabetes, cardiovascular diseases and certain types of cancers \citep{WHO2014}.
In fact, the risk of comorbidities increases with the increase in \gls{bmi}, where the risk becomes severe as the \gls{bmi} level approaches the obese category.

The number of overweight and obese population, in both adults and children, have risen in every country of the world, and the trend continues as we speak.
Estimated to account for 3.4 million deaths per year and 93.6 million \glspl{daly} in 2010, obesity is a serious disease that continues to grow in our society, which the world cannot ignore (citation from 2010 WHO report needed).

\subsection{Definition of obesity}
\label{sub:definition_of_obesity}
% TODO: add WHO website to the reference for the definition of obesity (or find proper articles)

Obesity is defined as an abnormal or excessive fat accumulation that may impair health of that individual.

One common and widely used approach to categorise individuals is by measuring the \gls{bmi} of the individual.
\gls{bmi} is a measurement based on the weight-to-height ratio of an individual that is used by clinicians and epidemiologists to classify adults into underweight, normal weight, overweight or obese categories.
The unit of \gls{bmi} is defined as the individual's weight in kilograms per square of the height in meters (kg/m$^2$).

\gls{who} (2014) have used \gls{bmi} measurements to define overweight and obesity in adults as an individual with \gls{bmi} $\geq$ 25 kg/m$^2$ and $\geq$ 30 kg/m$^2$, respectively.
For children under the age of 5, overweight and obesity were defined as weight-for-height greater than 2 standard deviation above \gls{who} Child Growth Standards median and greater than 3 standard deviation above the median, respectively.
For children aged between 5 to 19 years, overweight and obesity were defined as BMI-for-age greater than 1 standard deviation above \gls{who} Growth Reference Standards median and greater than 2 standard deviation above the median, respectively.

Although the use of \gls{bmi} as an accurate measure of body fat composition and/or representation of the individual's metabolic status remains controversial, many studies have used \gls{bmi} due to the ease of data collection compared to other measurements, such as [method] (citation needed).
Consequently, most of the publicly available clinical data have height and weight data from which one can calculate the patients' \gls{bmi}, and therefore able to collect the patients' obesity status more conveniently than, for example, waist-to-hip ratio.

\subsection{Prevalence of obesity}
\label{sub:prevalence_of_obesity}




% \subsection{Preventive measure of obesity}
% \label{sub:preventive_measure_of_obesity}



\subsection{Obesity and cancer risk}
\label{sub:obesity_and_cancer_risk}

\subsection{Impact of obesity}
\label{sub:impact_of_obesity}





\section{Cancer}
\label{sec:cancer}

% What is cancer.

\subsection{Hallmarks of Cancer}
\label{subsec:cancerhallmarks}

% TODO: edit the starting sentence so it's not a copy from the beginning
% It is now commonly understood that cancers are caused by dysregulation of various mechanisms or pathways that allow tumour cells to prolilferate, survive and migrate.
% Many of these dysregulated pathways frequently occur in cancer cells, and these characteristic traits, commonly known as the ``Hallmarks of Cancer'', were first described by \citet{Hanahan2000}.

% These hallmarks were initially presented in 2000 to provide a common framework for scientists to better understand cancer biology, and also to direct the focus of future research.
% After a decade of advances in cancer research, the ``Hallmarks of Cancer'' were revisited by Hanahan and Weinberg in 2011, and four more hallmarks were added on top of the six presented in the original paper in 2000 \citep{Hanahan2011}.

% Together, these ``Hallmarks of Cancer'' (fig 1) have provided a foundation of the general biological knowledge of cancer and its characteristics that allowed the researchers to better understand the underlying cancer biology.

% Although much reseach has been conducted on the ``Hallmarks of Cancer'', the molecular processes underlying these hallmarks have still not been fully explained \citep{Hanahan2011}.
% This is due to the fact that cancer cells develop progressively under selective pressures, and are also affected by many environmental factors \citep{Gatza2011}, which makes it difficult to identify causal factors of cancer development.
% Furthermore, these external factors affect how the cancer cells respond to the environment, and may also cause cancer progression and/or drug resistance.

\subsection{Tumour heterogeneity}
\label{sub:tumour_heterogeneity}




\subsection{Relationship between obesity}
\label{subsec:obsbackground}

% Why is it important in obesity context.

% The exact mechanism in which obesity accelerates cancer progression is unknown, but obesity is thought to disrupt the levels of various hormones (such as insulin, leptin, adeponectin, and other adipokines) that eventually activate tumorigenic pathways (reference the overview on obesity and tumour resistance).

\section{Genetic signatures}
\label{sec:genetic_signatures}

\subsection{Microarray technology}
\label{subsec:microarray_technology}

\subsection{Next generation sequencing technology}
\label{sub:next_generation sequencing_technology}



% Due to advances in sequencing technology over the last few years, many researchers have started to focus on bioinformatics based analyses of cancer cells to better understand the biology of cancer.
% Advantages of this strategy are that very specific gene expression and/or mutational signatures for certain cancer subtypes can be identified from the analyses, which the researchers and clinicians can study in much greater depth and details.

% TODO: have a leading sentence that introduces the gene signatures

\section{Obesity associated genetic signatures}
\label{sec:obesity_associated_genetic_signatures}

\subsection{\citet{Creighton2012} study}
\label{sub:creighton_study}



\subsection{\citet{Fuentes-Mattei2014} study}
\label{sub:fuentes_mattei_study}




\section{Pathway associated genetic signatures}
\label{sec:pathway_associated_genetic_signatures}

% Talk about \citet{Gatza2010a}

\subsection{\citep{Gatza2011} study}
\label{sub:gatza_study}





\section{Aim of the project}
\label{sec:aim}

This research aims to determine whether gene expression signatures exist  that are specific to obesity across multiple cancer types, and to investigate whether there are any common pathways being dysregulated in cancers based on these genetic signatures.
Better understanding of the pathways being dysregulated in cancer cells in obese patients may lead to improved clinical decisions, and contribute towards personalised treatment in the future.

