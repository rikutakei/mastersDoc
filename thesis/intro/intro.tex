\chapter{Introduction (draft)}
\label{ch:intro}

% TODO: also have a look at the masters talk write up from 2015

% This is taken from the committee report (I think it's a good intro to the intro)
% TODO: past tense...??
It is now commonly understood that cancers are caused by dysregulation of various mechanisms or pathways (known as the ``Hallmarks of Cancer'') that allow tumour cells to prolilferate, survive and migrate \citet{Hanahan2011}.
There have been many reports on the association between obesity and worsened cancer prognosis, suggesting that there may be distinct changes that underlie tumorigenesis and/or cancer progression that are specific to patients with higher \gls{bmi}.
The exact mechanism in which obesity accelerates cancer progression is unknown, but obesity is thought to disrupt the levels of various hormones (such as insulin, leptin, adeponectin, and other adipokines) that eventually activate tumorigenic pathways (reference the overview on obesity and tumour resistance).

% TODO: in the aims section, or here?
This research aims to determine whether gene expression signatures exist that are specific to obesity across multiple cancer types, and to investigate whether there aare any common pathways being dysregulated in cancers based on these genetic signatures.
Better understanding of the pathways being dysregulated in cancer cells in obese patients may lead to improved clinical decisions, and could thus contribute towards personalised treatment in the future.


\section{Obesity}
\label{sec:obesity}

General intro to obesity.
Implications of obesity -- disease associated with it and/or cancer.

\section{Cancer}
\label{sec:cancer}

What is cancer.

\subsection{Hallmarks of Cancer}
\label{subsec:cancerhallmarks}

What is it caused by (10 hallmarks of cancer).

\subsection{Relationship between obesity}
\label{subsec:obsbackground}

Why is it important in obesity context.

\section{Obesity associated genetic signatures}
\label{sec:obsgene}

Literature review, focussing on \citet{Creighton2012} and \citet{Fuentes-Mattei2014}.

\section{Aim of the project}
\label{sec:aim}

