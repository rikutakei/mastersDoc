\chapter{Introduction (draft)}
\label{ch:intro}

\section{Obesity}
\label{sec:obesity}

Obesity has been one of the major global problems for more than a decade, associated with many noncommunicable diseases such as diabetes, cardiovascular diseases and certain types of cancers \citep{WHO2014}.
In fact, the risk of comorbidities increases with the increase in \gls{bmi}, where the risk becomes severe as the \gls{bmi} level approaches the obese category \citep{WHO2000}.

The number of overweight and obese people, in both adults and children, have risen in every country of the world, and the trend continues as we speak.
Estimated to account for 3.4 million deaths per year and 93.6 million \glspl{daly} in 2010, obesity is a serious disease that continues to grow in our society \citep{Lim2012}.

\subsection{Definition of obesity}
\label{sub:definition_of_obesity}

Obesity is defined as an abnormal or excessive fat accumulation that may impair health of that individual \citep{Garrow1988}.
One common and widely used approach to categorise individuals is by measuring the \gls{bmi} of the individual.
\gls{bmi} is a measurement based on the weight-to-height ratio of an individual that is used by clinicians and epidemiologists to classify adults into underweight, normal weight, overweight or obese categories.
The unit of \gls{bmi} is defined as the individual's weight in kilograms per square of the height in meters (kg/m$^2$).

\gls{who} (2014) have used \gls{bmi} measurements to define overweight and obesity in adults as an individual with \gls{bmi} $\geq$ 25 kg/m$^2$ and $\geq$ 30 kg/m$^2$, respectively.
For children under the age of 5, overweight and obesity were defined as weight-for-height greater than 2 standard deviation and 3 standard deviation above \gls{who} Child Growth Standards median, respectively.
For children aged between 5 to 19 years, overweight and obesity were defined as BMI-for-age greater than 1 standard deviation and 2 standard deviation above \gls{who} Growth Reference Standards median, respectively.

Although the use of \gls{bmi} as an accurate measure of body fat composition and/or representation of the individual's metabolic status remains controversial, many studies have used \gls{bmi} due to the ease of data collection compared to other measurements, such as waist-to-height ratio \citep{Lee2008, Gelber2008}.
Consequently, most of the publicly available clinical data have height and weight data from which one can calculate the patients' \gls{bmi}, and therefore able to collect the patients' obesity status more conveniently than, for example, waist-to-height ratio.

\subsection{Prevalence of obesity}
\label{sub:prevalence_of_obesity}

From the latest global status report on noncommunicable diseases by \citet{WHO2014}:
\begin{quote}
	\textit{
		In 2014, 39\% of adults aged 18 years and older (38\% of men and 40\% of women) were overweight.
		The worldwide prevalence of obesity nearly doubled between 1980 and 2014.
		In 2014, 11\% of men and 15\% of women worldwide were obese.
		Thus, more than half a billion adults worldwide are classed as obese.
	}
\end{quote}

\noindent
This equated to approximately 1 in every 14 people in the world classed as obese in 2014, and therefore, 1 in 14 people around the world were also at severe health risks associated with obesity.
It was also noted in the report that women were more likely to be obese than men, and that the prevalence of overweight and obesity increased as the income level of the countries increased \citep{WHO2014}.

In addition to this, the worldwide prevalence of childhood overweight and obesity has been increasing steadily since 2000 \citep{WHO2014}.
Furthermore, the prevalence of overweight in children under 5 years was estimated to reach from 6.3\% in 2013 to 11\% by 2025 if the current trend continued \citep{WHO2014}.
The current obesity status of the adult population is bad already, but the fact that more of the younger population who are overweight and obese poses a much greater risk for the health of the population.

% TODO: Add western-specific and NZ specific statistics on obesity
% TODO: Add male/female specific stats
% TODO: Add Maori/PI specific stats

% Until these terrible trends stop, and even after the trends stop, the world will need a plan to live along with this ongoing global epidemic of obesity and its associated diseases.

\subsection{Physiological mechanism of obesity}
\label{sub:physiological_mechanism_of_obesity}

Obesity is caused by continuous intake of food without expending all of the energy gained.
The key to the problem is the control of food intake and energy expenditure, and how they are regulated in your body.
The role of hypothalamus is central to the control of satiety and many mutations that cause obesity affects the satiety signals to and/or from hypothalamus.
The hypothalamus receives both nueronal and hormonal inputs from the body and propagates the signal to the downstream neurons that ultimately affect the food intake of the individual \citep{Bell2005, Spiegelman2001}.

There are two types of neurons in the arcuate nucleus (located in the hypothalamus): orexogenic and anorexogenic neurons.
Orexogenic neurons are responsible for the promotion of food intake and reducing energy expenditure, whereas the anorexogenic neurons have the opposite effect \citep{Barsh2002}.
When stimulated by, for example, endocrine signals, these two different types of neurons act on the downstream neurons to promote or reduce food intake and energy expenditure.

Leptin-melanocortin pathway is the main pathway in which the level of satiety is controlled and regulated \citep{Spiegelman2001}.
In this pathway, both the orexogenic and anorexogenic neurons (located in the arcuate nucleus of the hypothalamus) have crucial roles in controlling the satiety of the individual, and therefore their eating behaviour \citep{Barsh2002,Bell2005}.
When a satiety-controlling hormone such as leptin reaches the arcuate nucleus region of the hypothalamus, it can stimulate or inhibit orexogenic, anorexogenic, or both orexogenic and anorexogenic neurons.
Depending on which neurons are stimulated or inhibited, the hormone will ultimately alter the satiety level of the individual.

Though there are many hormones that regulate satiety, only leptin and insulin will be covered here.
Leptin gene was first discovered by \citet{Zhang1994} and the discovery of the leptin receptor gene was followed shortly after by \citet{Tartaglia1995}.
Together, the identification of the leptin and leptin receptor genes was a major break through in the field of obesity research.

Leptin is a hormone that signals satiety and reduces food intake, and is secreted mainly by the white adipose tissue \citep{Moustafa2013,Zhang1994}.
The circulating concentration of leptin in the body is proportional to the body fat stores \citep{Barsh2002, Moustafa2013}.
Leptin signals through the \gls{jak}/\gls{stat} pathway via the long form of the leptin receptor that is present in high expression in the hypothalamus \citep{Ghilardi1996,Lee1996}.
Depending on the type of neurons the leptin receptor is expressed, leptin stimulate or inhibit these neurons and ultimately produces an anorexogenic signal to the downstream effector neurons \citep{Bell2005}.

Insulin, perhaps the most well-known hormone for its role in glucose homeostasis, also has an important role in satiety and food intake regulation.
Briefly, when insulin binds to the insulin receptor, \gls{irs} protein family is recruited and phosphorylated by the insulin receptor's intrinsic tyrosine kinase activity \citep{Saltiel2002}.
The phosphorylated \gls{irs} protein then activates \gls{pi3k}, which in turn activates its downstream targets, causing a rapid signal cascade through the cell \citep{Saltiel2002}.

The evidence of insulin's involvement in the satiety control was not particularly clear even though there were strong evidence of obesity with neuron-specific deletion of insulin receptor, and that the insulin level, like leptin, was proportional to the fat mass also suggested the role of insulin in satiety and body weight control \citep{Barsh2002, Bruning2000, Woods1979}.
This was perhaps due to the fact that leptin and insulin used two different signalling pathways in the cell; leptin used the \gls{jak}/\gls{stat} signalling pathway, whereas insulin used the \gls{pi3k} pathway for signalling \citep{Ghilardi1996}.
This was puzzling to the researchers as there seemed to be no converging point of the two hromones in these pathways.

However, leptin, and subsequently insulin, was shown to activate ATP-dep\-endent K$^+$ channel in glucose-responsive neurons in the hypothalamus \citep{Spanswick1997, Spanswick2000}.
The activation of K$^+$ channel provided evidence for the two pathways to act similarly on the cell, at least for an acute response, suggesting a convergence in the pathway response by the two hormones, and therefore their role in satiety control.

Satiety control is crucial in order to manage  and maintain the body weight in a healthy state.
As described above, there are many components that regulate satiety and disruption of any part of this satiety controlling pathway can encourage the development of obesity.

\subsection{Factors that affect the prevalence of obesity}
\label{sub:factors_that_affect_the_prevalence_of_obesity}

As one can suspect, there are many factors involved with the likelihood of an individual to become overweight and obese.

\subsubsection{Individual-level (micro-level) factors}
\label{ssub:Individual-level_(micro-level)_factors}

There are many environmental factors associated with the global rise in obesity.
Even though these environmental factors are probably the most significant contributor to the current global obesity epidemic, there are some individual-level, genetic factors that contribute to the overall increase in the proportion of the population that are obese.

There is much evidence of obesity being caused by individual-level genetic factors that make some individuals more predisposed to obesity than others.
Generally speaking, there are two classes of genetically-driven obesity: monogenic obesity and polygenic obesity \citep{Moustafa2013}. \\

\noindent
Monogenic obesity is where a single mutation in one of the genes that is directly involved in the satiety controlling pathway causes severe obesity in that individual.
Since these mutations are usually directly related to the key constituents in the pathway mentioned earlier (\cref{sub:physiological_mechanism_of_obesity}), the resulting phenotype from the mutation causes severe early-onset obesity.
However, although these mutations cause a devastating effect on the individual carrying the mutation, these mutations occur very rarely in the population.
% With that said though, the identification of these mutations have shed light on the mechanisms in which obesity is developed and it is worth mentioning some of these mutations to highlight the importance of the components involved in the obesity pathway.

An example of a mutation that causes severe early-onset obesity is a mutation in the leptin gene.
\citet{Montague1997} found a single nucleotide deletion mutation at position 133 of the leptin gene in children with early-onset obesity.
They further showed that the mutation caused a frameshift mutation in the protein, resulting in an inactive form of leptin that was not able to be secreted \citep{Montague1997}.
As a result, children with this mutation was constantly hungry due to the lack of anorexogenic signals, and became obese at a very early stage of their lives.
\\

% In close relation to this mutation is the mutation in the leptin receptor gene.
% \citet{Clement1998} have discovered a single nucleotide substitution mutation that led to the leptin receptor transcript that skipped exon 16, leading to a form of the receptor with no transmembrane or intracellular domain.
% Individuals with this rare mutation showed extreme hyperphagia and symptoms of early-onset obesity, even though their circulating leptin levels were present abundantly \citep{Clement1998}.

% The next couple of mutations are associated with the synthesis and the processing of \gls{pomc}.
% \citet{Krude1998} found single nucleotide mutations in the exonic region of the \gls{pomc} gene that affected the transcription of the $\alpha$-\gls{msh} region (and some other regions) of \gls{pomc}.
% \citet{Jackson1997} identified a mutation in the \gls{pc1}, an enzyme responsible for the processing of many prohormones and neuropeptides, including \gls{pomc}.
% In either case, the mutation resulted in the lack of $\alpha$-\gls{msh} in the neurons for satiety signalling and caused severe early-onset obesity.

% As mentioned earlier, even though these mutations are very rare and affects only those individuals that have one of these mutations, it had a profound impact in the field of obesity research and contributed significantly to the knowledge of the mechanism of how obesity develops.

\noindent
Polygenic obesity is where a combination of variants in multiple genes or multiple regions of the genome, together with environmental factors, causes obesity in the individual \citep{Moustafa2013}.
Unlike in monogenic obesity where a single mutation in a gene causes severe obesity, the variants in some of the genes or genomic regions involved in polygenic obesity have little effect on the individual's obesity status by itself, but in combination with many other variants and together with environmental factors, it has a serious obesogenic effect on the individual.

% Known as the ``thrifty gene'' hypothesis, \citet{Neel1962} suggested that the genes that were important in obesity have been around in the population for many generations, as it would have had a selective advantage in an environment where the population experienced starvation more frequently than the current generation.
% However, as food became more accessible, a subset of the population who were particularly good at storing fat and survive the starvation period ``overreacted'' to this rapid change in the environment \citep{Bell2005, Spiegelman2001}.
% These observations in the rise in obesity in subsets of population cannot be explained simply by external factors alone, and there is a significant genetic component involved in this phenomenon \citep{Bell2005}.

There have been many studies that have investigated the association of genes and/or genomic regions with obesity.
Of those, many of the \glspl{gwas} published in 2007 that showed strong evidence of the \gls{fto} gene with \gls{bmi} and obesity were of greatest significance \citep{Dina2007,Frayling2007,Gerken2007,Scuteri2007}.
Until these studies came out, there has been no common genes or genomic regions that associated with obesity that was reproducible and reliable \citep{Frayling2007}.
Thus, these studies were the first evidence of common variants that contributed to the polygenic obesity phenotype in the population.

% Discovery of the \gls{fto} gene has lead to extensive research on the gene, and the mechanism of action by the \gls{fto} variant has been recently uncovered  from these efforts \citep{Claussnitzer2015a,Smemo2014a}.
% The association between obesity and genetic variations, and their mechanisms that lead to obesity, has not been fully elucidated due to its complexity, but it may not be long before they are fully uncovered.

\subsubsection{Population-level (macro-level) factors}
\label{ssub:Population-level_(macro-level)_factors}

Even though some individual-level factors are involved with obesity, it is clear that these individual-level factors cannot be accounted for the global epidemic of obesity that the world currently observes.
For this reason, obesity is thought to be caused mainly by the environmental factors that act at a population-level.\\

\noindent
The first key factor that contirbutes to the global rise in the proportion of the population that is obese is the availability and abundance of certain food in the country.
With the help from government trade policies, food and other goods have become easier to trade between the country, helping the economic growth of many countries around the world \citep{Kearney2010}.
This led to greater abundance of food in general, greater number of choices of food and shifted the overall nutritional status of the country \citep{Malik2013}.
This shift in the food choice and availability (and the economic benefit associated with the trades) had a positive impact on the country, but the growth in the food market also resulted in population-wide rise in obesity.
% This shift in the food choice and availability (and the economic benefit associated with the trades) had a positive impact on the country, but the growth in the food market also resulted in a different, much larger problem -- obesity.

As an example, in the \gls{usa}, the cost of corn and soy is low due to the fact that they are the raw ingredients for most processed food and beverages, such as high fructose corn syrups used to sweeten many soft drinks in the \gls{usa} \citep{Malik2013}.
Moreover, corn and soy are also the main food for the livestock, which results in lower prices for meat \citep{Malik2013}.
On the other hand, the prices of fruits and vegetables remain expensive due to the lack of support by the government to lower the cost associated with the production \citep{Malik2013}.

As a result, more people are inclined to consume food that are cheap, have little nutritional value and high in energy, than the food that are expensive, nutritional and healthy alternatives.
Thus, the population is more likely to become obese by making poor food choice.\\

\noindent
The second factor to consider is the behavioural changes led by the urbanisation of the population.
% Urbanisation is defined as the gradual increase in the proportion of people living in urban areas.
As more people take up urban lifestyle, they are exposed to the new environment where there is a better range of food selection, as well as technological and mechanical advancement in transport and other daily chores that may not be available in rural areas.

Although there are many advantages involved with urbanisation, such as access to developed health care systems, education and advanced technologies, the lifestyle change also impose negative health behaviour that ultimately lead to positive energy balance and therefore obesity \citep{Malik2013}.
Lack of physical activity is one of the obvious components that is linked with the urbanisation of the population.
% Currently, the recommended amount of physical activity to keep an individual fit and healthy is $\geq$ 150 min of moderate-intensity aerobic physical activity per week, or $\geq$ 30 min of physical activity on most days of the week \citep{Pate1995, WHO2010}.
In many urban cities, the recommended level of physical exercise is difficult to achieve and maintain due to the shift towards the sedentary behaviours that are encouraged by the developed transport system, automated household chores, and indoor entertainment \citep{Malik2013}.
\\
% Urbanisation by itself cannot be accounted for the lack of exercise in the population, but it is true that urbanisation makes the bar higher for many people to take up the recommended level of fitness and maintain it. \\

\noindent
The last factor to be mentioned is the income and socioeconomic status of the population.
As the average income increases, more people are able to buy food and are likely to take up the sedentary lifestyle.

From this, one would expect the high-income group to have the largest impact from obesity, but on the contrary to this expectation, the biggest impact from obesity is seen in the low- and middle-income groups of the population \citep{Malik2013}.
Access to the developed health care system will allow weight maintenance of the population, but this is likely to be of most benefit for the high-income group in the population.
The reason for this, quite simply, is the cost associated with the visit to the health care system, as high-income group is more likely to make a consistent and periodic visit than the low or middle-income group of the population \citep{Malik2013}.

So, for the low- and middle-income groups, the rise in average income in these groups allow the people to have access to food and activities that promotes sedentary bahaviours, but the limited access to the health care system, education and recreational activities that promote physical activites ultimately lead to positive energy balance \citep{Malik2013}.
In contrast, the high-income group of the population have better access to the facilities that promote weight maintenance, thus have less of an impact compared with the low- and middle-income groups of the population.
\\

% \noindent
% The last factor, which ties in strongly with the first key factor, is the choice of diet made by the population.
% The amount of low-quality food have increased significantly in many rapidly growing countries, as these high-energy, low-nutritional food are cheaper and more profitible than the natural produce that have higher nutritional value \citep{Kearney2010}.
% As a result, more people tend to purchase these low-quality food more often, and consume more energy than they are required.

% However, many people eat responsibly and avoid these low-quality diet and prevent overconsumption of such food, but the overconsumption of these food is not a big problem as such, rather the overall quality of the food \citep{Malik2013}.
% These food are usually high in refined sugar and unhealthy fat, highly processed, contain very little nutritional value and are low in fibre.
% As a result, consumption of these food increases the overall energy intake and achieve no nutritional benefits from the food.

% Whether it is by choice or not, consumption of these low-quality diet leads to obesity and obesity-related diseases. \\

\noindent
All of these population-level environmental factors in synergy with the individual-level genetic factors have contributed to the global epidemic of obesity.

% \subsection{Impact of obesity}
% \label{sub:impact_of_obesity}

% As mentioned at the beginning of the current section, obesity has a devastating effect on the health of the population in a global scale.
% Overweight and obesity have been a risk factor for many diseases, including type 2 diabetes, cardiovascular diseases, hypertension, and many types of cancers \citep{Franks2010, Guh2009, Yach2006}.
% As \citet{Yach2006} mentions in their paper, ``overweight and obesity have become to diabetes what tobacco is to lung cancer'', and it is important to understand that obesity has significant risks and comorbidities associated with it.

% Although adulthood obesity is clinically important for risk assessment of diseases, childhood obesity has been focussed more heavily in recent years,  as it has greater public health implications.
% There are many immediate consequences that are related to childhood obesity, but the diseases that develop as these children grow up are more critical than the earlier effects.
% A paper by \citet{Must1999} reviewed many of the clinical studies done in the last several decades and summarised the risks associated with childhood obesity.
% Among the many risks associated with obesity, obese children, compared to lean children, had over 8.5-fold increased risk of developing hypertension as adults \citep{Must1999}.
% On top of this, the relative risks for all-cause mortality and coronary heart disease-specific mortality were 1.5 and 2.5, respectively \citep{Must1999}.

% In a more recent study by \citep{Franks2010}, they have shown that childhood obesity (as well as glucose intolerance and hypertension in childhood) was strongly associated with premature deaths from endogenous causes in American Indian population from Arizona (incidence rate ratio of 1.4 per unit of \gls{bmi} z-score).

% Health related consequences of obesity are clear from many epidemiological studies, but it is noteworthy that obesity also has an economical impact on the population as well.
% In the \gls{usa}, within the span of 5 years, the medical cost due to diabetes have risen from \$44 billion to \$92 billion \citep{Yach2006}.
% Although not all diabetes are directly caused by obesity, obesity would have assisted in the rise in the costs associated with diabetes.
% In addition to this,
% The costs that are associated with the treatments of comorbidities of obesity affects the household incomes, increased premature retirement and unemployment, which could have a significant economical impact on the society over time \citep{Yach2006}.

% As more of the younger generation is gaining weight and becoming obese, having the risks and disease burden associated with obesity at such early age are not ideal (both economically and epidemiologically) and poses a threat to the growing population that are becoming more obese.

% \subsection{Preventive measures of obesity}
% \label{sub:preventive_measures_of_obesity}

% It is apparent from previous sections that obesity has become a serious problem that the world needs to prioritise to solve immediately.
% Even though obesity has been regarded as one of the many global problems to solve, the progress has been slow.

% \citet{Sacks2009} proposed an \gls{opa} framework that was based on the framework developed by \citet{WHO2006}.
% In this framework, \citet{Sacks2009} have added extra features so that the policies were targeted to impact three levels of the underlying environment of obesity: the socio-ecological environment (upstream), lifestyle/behavioural changes (midstream), and health services support (downstream).

% The upstream approach targets the underlying economic, social and physical norms that the population is currently in.
% By targeting the broad socioeconomic environment that the population lives in, such as taxation, education, employment and housing \citep{Sacks2009}.
% In doing so, it is possible to have a significant change in the population behaviours and eventually improve the health outcomes of the population.

% The midstream approach targets the population or subpopulation behaviour change to improve the eating and physical activity behaviours in these groups.
% Unlike in the upstream approach, the midstream approach aims to change the behaviour by implementing social marketing and programmes that encourage such changes \citet{Sacks2009}.
% Again, the result of this approach will be an improvement in the overall positive behaviour of the population towards healthy eating and physical activity.

% The downstream approach targets to support the health services and clinical interventions.
% Unlike upstream and midstream approaches, downstream approach generally acts on an individual level, whereby the focus is on the management of the health of an individual and/or the individual's family \citet{Sacks2009}.
% Although not as efficient as upstream or midstream approaches, downstream approach will help improve the habits and behaviours to promote weight management.\\

% \noindent
% In the recent years, obesity has been declared as a global priority \citep{Malik2013}.
% Furthermore, comprehensive global monitoring framework, multi-sectoral actions and strengthening of the national policies have been implemented to better control the obesity epidemic \citep{Malik2013}.
% In many countries, the nutritional labelling on products have been adjusted to include the ingredients that have negative impact on obesity (for example, \textit{trans} fat), and even reduce or remove these ingredients from food \citep{Malik2013}.

% Food marketing can also be a problem in a highly competitive society that we live in.
% To sell their products that are high in fat, sugar and portion size, food industries market and advertise their products to the population, especially young children, in order to stay ahead in their market place \citep{Chan2010, Malik2013}.
% Even though there is a strong evidence of food marketing influencing the eating behaviours and food preferences in children, very few countries have been able to reduce these marketing and advertising strategies of unhealthy food \citep{Hawkes2007a, Malik2013}.

% In contrast to these upstream/midstream approaches, there are some clinical interventions and treatments that help manage weight in an individual.
% One of these therapies is bariatric surgery for obese patients, especially morbidly obese patients \citep{Buchwald2004}.
% Through their meta-analysis, \citet{Buchwald2004} showed that bariatric surgery was able to reduce many of the the comorbidities associated with obesity.
% The fact that these improvements and complete resolution of the comorbidities in high proportion of patients that underwent bariatric surgery shows that, even though it acts on an individual level, downstream approaches such as bariatric surgery may be a highly effective, but not an efficient, method of preventing obesity in the population.\\

% \noindent
% It is evident that the most efficient way to reduce the proportion of the population that are obese is to implement national policies that affect the upstream or midstream factors.
% These policies, although they are efficient at targeting the underlying problem, may take a long time to make a significant effect.
% On the otherhand, individual therapeutic plans that affect the downstream factors may be very effective in achieving the desired goal of weight loss, but it is very resource intensive and upscaling this approach would be practically impossible.

% The management of obesity at an early stage with midstream and/or downstream approaches while the upstream approach takes effect would be the key to success in controlling for the global obesity epidemic.

\section{Cancer}
\label{sec:cancer}

% What is cancer.
\begin{quote}
	\textit{
	``Cancer is a generic term for a large group of diseases that can affect any part of the body \ldots{}
	One defining feature of cancer is the rapid creation of abnormal cells that grow beyond their usual boundaries, and which can then invade adjoining parts of the body and spread to other organs \ldots{}
	referred to as metastasizing.
	Metastases are the major cause of death from cancer.''
	\citep{WHO2016}
	}
\end{quote}

\subsection{Prevalence and mortality of cancer}
\label{sub:prevalence_and_mortality_of_cancer}

% TODO: Add western-specific and NZ specific statistics on cancer
% TODO: Add male/female specific stats
% TODO: Add Maori/PI specific stats

Cancer is considered as one of the leading causes of deaths around the world, accounting for 8.2 million deaths in 2012 \citep{WHO2014}.
More than half of cancer deaths are caused by lung, breast, colorectal, stomach and liver cancers \citep{WHO2014}.
The leading cause of deaths in high-income countries is lung cancer in both men and women, then colorectal cancer and breast cancer for men and women, respectively \citep{WHO2014}.
Although the levels of mortality vary among different countries, cervical, liver and stomach cancers make up a large proportion of cancer deaths among low- and middle-income countries \citep{WHO2014}.
\\

\noindent
\citet{Bray2013} estimated the global prevalence of cancer in adult population in 2008.
Since there is no clear definition for cancer prevalence, \citet{Bray2013} used \gls{ldp} as an estimate for cancer prevalence, and measured the incidence of cancer during the period between 2004 and 2008 (5 years).
% measuring the incidence of cancer during the period between 2004 and 2008 (5 years) who were still alive at the end of 2008.
From these estimates, Eastern Asia had the greatest 5-year prevalence of cancer, with approximately 7 million people with cancer in 2008, followed by North America and Western Europe with approximately 4.7 and 3 million people with cancer \citep{Bray2013}.
When the population of these regions was taken into account and the proportion of cancer prevalence calculated, Western Europe had the highest prevalence proportion of 1.9\%, followed closely by Australia/New Zealand with 1.82\% and North America with 1.7\% \citep{Bray2013}.

Looking at the prevalence of cancer by specific types, breast cancer was the most prevalent type of cancer with 5.2 million cases in 2008 \citep{Bray2013}.
Colorectal and prostate cancers were second and third most prevalent cancer types, with approximately 3.2 million cases for  both of them \citep{Bray2013}.
\citet{Bray2013} noted that female-specific cancer types were the most prevalent types of cancer in 75\% of the countries in the world.
Furthermore, the prevalence of breast, cervix or prostate cancer is the highest in almost 95\% of the countries \citep{Bray2013}.
Although these estimates provide a relatively crude measure of cancer prevalence, \citet{Bray2013} provided a useful overview of what the prevalence of cancer is like around the world.\\

\noindent
The total number of deaths by non-communicable diseases (including cancer) is predicted to increase, reaching 52 million deaths by 2030 \citep{WHO2014}.
With the prevalence of cancer at such high levels around the world, and with cancer mortality increasing in the future, cancer is a serious burden that the world has to face.

\subsection{Mechanisms of cancer development}
\label{sub:mechanisms_of_cancer_development}

It is now commonly understood that cancers are caused by dysregulation of various molecular mechanisms or pathways that allow tumour cells to prolilferate, survive and migrate.
By taking advantage of these pathways, cancer cells are able to grow, and eventually metastasise to other parts of the body.
The environment in which the tumours arise can also be an important factor for their evolution and growth, contributing to the heterogeneity of the cancer cells within the tumour or between tumours at different sites of the body.

\subsubsection{Hallmarks of Cancer}
\label{subsubsec:cancerhallmarks}

Extensive research in cancer biology have been able to uncover many pathways and regulatory mechanisms involved in cancer.
% , so much that there were too many possibilities in which cancer could develop from.
With more than 100 distinct types and subtypes of cancers, each with its own distinct development mechanisms, it became difficult to focus on the overarching mechanisms of cancer development \citep{Hanahan2000}.
To resolve this, \citet{Hanahan2000} proposed ``six essential alterations'' that cancer cells usually take advantage of during development and progression.
These six key mechanisms were coined as being ``The Hallmarks of Cancer'', and have been used extensively by many researchers as their guide to target their research in specific areas of cancer biology.

Ten years after the first review where the concept of Hallmarks of Cancer was proposed, Hanahan and Weinberg revisited these hallmarks and added four more on top of the six hallmarks presented in 2000 \citep{Hanahan2011}.
The ten Hallmarks of Cancer are: sustaining proliferative signalling, evading growth suppressors, enabling replicative immortality, activating invasion and metastasis, inducing angiogenesis, resisting cell death, avoiding immune destruction, tumour-promoting inflammation, genome instability and mutation, and deregulating cellular energetics (\cref{fig:hallmarks}).
Each of these hallmarks will be described briefly to provide an overview of the significance of these mechanisms in tumour biology.

\begin{figure}[h!]
	\centering
	\includegraphics[width=0.8\linewidth]{intro/hallmarks}
	\caption[Hallmarks of Cancer]{Hallmarks of Cancer. The ten Hallmarks of Cancer as proposed by Hanahan and Weinberg in 2011. (Figure adapted from \citet{Hanahan2011})}
	\label{fig:hallmarks}
\end{figure}

\paragraph{Sustaining proliferative signalling}

\noindent
Cancer is a disease of uncontrolled cell proliferation and division.
Normal cells cannot proliferate without any external mitogenic growth signals, but cancer cells are able to deregulate these signals to maintain a steady proliferative signal \citep{Hanahan2011}.

There are many ways in which cancer cells can maintain this signal.
Firstly, cancer cells can acquire the ability to produce growth factors and express the corresponding receptor, creating a positive feedback mechanism that continuously stimulates the cells to proliferate \citep{Hanahan2000}.
Secondly, they can express more receptors for a given growth signal, and thus become ``hyperresponsive'' to the signal \citep{Hanahan2000,Hanahan2011}.
Thirdly, cancer cells can stimulate the neighbouring normal cells within the tumour microenvironment to produce more growth factors that further assist them to grow \citep{Bhowmick2004, Liotta2001, Wiseman2002}.
Finally, cancer cells are able to maintain proliferative signalling by mutating the growth factor receptor itself, or the downstream signalling targets of the growth factor receptor pathway \citep{Fuqua1991,SuHuang1997,Satyamoorthy2003}.
By maintaining the the proliferative signals, cancer cells are able to continuously grow in their environment.

\paragraph{Evading growth suppressors}

\noindent
As well as maintaining proliferative signals, cancer cells must also evade signals that inhibit cell growth.
p53 protein is a tumour suppressor protein that has a  role in cell cycle arrest \citep{Hanahan2011,Levine1997}.
As \citet{Hanahan2011} mention in their review, ``TP53 receives inputs from stress and abnormality sensors that function within the cell's intracellular operating systems''.
The level of expression and its stability is crucial for the function of p53 protein, as the abundance of p53 protein in the cell triggers cell cycle arrest or cell \gls{apoptosis}, depending on the extent of the cell signal \citep{Fridman2003,Hanahan2011,Levine1997}.
When abundant and active, p53 protein transcribes the p21 protein which, through series of events in the p16-cyclin D$_1$-cdk4-\gls{rb} pathway, triggers cell cycle arrest \citep{Levine1997}.
In many cancers, p53 protein is usually mutated and selected against to allow continuous growth of the tumour cells.

% Two tumour suppressor proteins have an essential role in this evasion: the \gls{rb} and p53 proteins.

% \Gls{rb} protein is a known tumour suppressor protein that is important in cell cycle regulation \citep{Burkhart2008,Hanahan2011}.
% \Gls{rb} protein is known to arrest the cells in \gls{g1} phase of the cell cycle by regulating E2F transcription factors in the cell, in reponse to molecules such as \gls{tgfb} \citep{Burkhart2008,Hanahan2000}.
% Although \gls{rb} protein is known for its role as a checkpoint in \gls{g1} phase, it also takes part in the regulation of apoptotic cell death, cell differentiation and chromosome stability \citep{Burkhart2008}.
% The contribution of the loss of function of the \gls{rb} protein in tumorigenesis and tumour progression is evident, but at which stage of the development this loss of function mutation is crucial for the cancer cells is yet to be clarified \citep{Burkhart2008}.

% p53 protein is another tumour suppressor protein that is closely related to the \gls{rb} protein and its role in cell cycle arrest \citep{Hanahan2011,Levine1997}.
% As \citet{Hanahan2011} mentions in their review, ``\gls{rb} transduces growth-inhibitory signals that originate largely outside of the cell, TP53 receives inputs from stress and abnormality sensors that function within the cell's intracellular operating systems''.
% The level of expression and its stability is crucial for the function of p53 protein, as the abundance of p53 protein in the cell triggers cell cycle arrest or cell \gls{apoptosis}, depending on the extent of the cell signal \citep{Fridman2003,Hanahan2011,Levine1997}.
% When abundant and active, p53 protein transcribes the p21 protein which, through series of events in the p16-cyclin D$_1$-cdk4-\gls{rb} pathway, triggers cell cycle arrest \citep{Levine1997}.

\paragraph{Resisting cell death}

\noindent
\Gls{apoptosis}, the natural and controlled death of a cell, is one of many defence mechanisms in the body to prevent the development of cancer.
The triggering of \gls{apoptosis} is controlled and maintained by the balance between the pro- and anti-apoptotic signals that affect the downstream apoptotic proteins \citep{Hanahan2011}.

% \Gls{bcl2} family proteins are the major players of apoptosis.
% Embedded in the mitochondrial membrane, the Bax subfamily of \gls{bcl2} proteins are pro-apoptotic, and are under constant repression by the anti-apoptotic proteins of the \gls{bcl2} subfamily proteins \citep{Adams2007,Hanahan2011}.
% The inhibition of the pro-apoptotic proteins are achieved by the interaction between the pro- and anti-apoptotic proteins via the \gls{bh3} domain \citep{Adams2007}.
% In addition to the pro-apoptotic Bax subfamily proteins, there is a family of proteins that contain a single \gls{bh3} domain, and hence named \gls{bh3}-only family proteins \citep{Adams2007}.
% \gls{bh3}-only proteins either disrupt the inhibition of the Bax family proteins by the \gls{bcl2} subfamily proteins, or directly stimulate the Bax subfamily proteins \citep{Adams2007}.
% The activation of the pro-apoptotic proteins disrupt the mitochondrial cell membrane adn releases cytochrome \textit{c}, which in turn triggers the activation of the apoptotic proteases caspase 8 and caspase 9, resulting in apoptosis \citep{Adams2007,Hanahan2011}.

The precise cellular conditions that trigger apoptosis are yet to be uncovered, but there is a clear evidence of p53 tumour suppressor playing a key role in apoptosis initiation.
When \acrshort{dna} damage is sensed, p53 protein transcribes the pro-apoptotic members of the \gls{bcl2} family proteins, such as Bax, Puma and Noxa \citep{Fridman2003,Hanahan2011}.
The fact that p53 protein is involved not only in cell cycle regulation, but also in apoptotic pathway shows its importance in the control of cancer development, and perhaps that is the reason why more than 50\% of tumours have a mutation in the \textit{TP53} gene \citep{Levine1997}.

\paragraph{Enabling replicative immortality}

\noindent
When acquired by cancer cells, the previous three hallmarks allow the cells to continuously grow in an uncontrolled manner.
However, this is not the case as the disruption in cell signalling on its own does not trigger the rapid expansion of the tumour -- only when the cells have the capability of replicating infinitely \citep{Hanahan2000, Hanahan2011}.

There are two barriers in achieving the infinite replicative capacity: senescence and crisis \citep{Hanahan2011}.
When cells are grown in a culture, cells are able to replicate and grow until a certain point, and the cells reach the stage of senescence where the cells stop growing but are still viable \citep{Hanahan2011}.
The cells that manage to bypass this senescence phase by mutating the tumour suppressor proteins such as p53 protein, undergoes the crisis phase where the majority of the cells in the culture dies \citep{Hanahan2011}.
Very rarely, some cells are able to survive the crisis phase and acquire the ability to replicate infinitely, and this transition has been termed ``immortalisation'' \citep{Hanahan2011, Wright1989}.
Without immortalisation, cancer cells are not able to generate a macroscopic tumours that are capable of killing the host organism \citep{Hanahan2000,Hanahan2011}.

\paragraph{Inducing angiogenesis}

\noindent
Like normal cells, cancer cells also require access to oxygen and nutrient supply from the blood.
However, new blood vessel formation, or angiogenesis, only occur in a handful of occasions, and thus the tumour needs to acquire the ability to stimulate angiogenesis in the microenvironment in which they live in \citep{Hanahan2011}.

It is thought that the angiogenic ability of the tumour is gained at an early stage of tumour development by disrupting the ``angiogenic switch'' \citep{Hanahan2011}.
% Like many cellular processes, angiogenesis is controlled by the balance between the inducing and inhibiting agents of angiogenesis.
\Gls{vegfa} is the major growth factor that triggers angiogenesis by signalling through the \acrshort{vegf} receptors (though some receptors have inhibitory effect) \citep{Yancopoulos2000}.
Opposing the effect of \Gls{vegfa} is the \gls{tsp1}, a protein that is known to inhibit the process of angiogenesis by affecting the bioavailability of \acrshort{vegf}, as well as promoting apoptosis \citep{Kazerounian2008}.
In general, many cancers stimulate the expression of \gls{vegfa} protein and reduces the expression of \gls{tsp1} \citep{Kazerounian2008}.

\paragraph{Activating invasion and metastasis}

\noindent
One of the prime reasons that cancers are so difficult to treat and have high mortality is due to their ability to metastasise to other parts of the body.
Once metastasised to different parts of the body, the clinician not only has to consider the primary tumour site, but also the second, third, or even fourth tumour sites, which may or may not have similar tumour biology to the primary tumour, and thus have to consider multiple treatment plans to cure the patient.

Metastasis is a multi-step process that requires the selection and survival of cancer cell(s) that is able to metastasise (``seed''), and is also compatible with the specific target tissue where they prosper (``soil'') \citep{Talmadge2010}.
% The downregulation or loss of function of E-cadherin, a protein important in cell-to-cell connection and adherence, is known to trigger tumour invasion and metastasis through the reversible process of \gls{emt} \citep{Hanahan2011,Kalluri2009}.
Once mobile and invasive, metastatic cancer cell is able to circulate in the vascular system and eventually invade the target organ, where it must undergo \gls{met} and colonise the site \citep{Hanahan2011,Kalluri2009}.
There are numerous signals, growth factors and pathways that initiate the metastatic phenotype in cancer cell, and it remains an active field of research \citep{Hanahan2011,Kalluri2009}.

\paragraph{Genome instability and mutation}

\noindent
Many of the hallmarks are acquired by the cancer cells during tumour growth, through many genomic changes and mutations of the essential genes that allow cancer cells to survive, proliferate, invade and metastasise.
Thus, it is only logical to induce that many, if not all, cancers are more prone to genetic mutations and restructuring, which allows the cancers to subsequently acquire the already described Hallmarks of Cancer \citep{Hanahan2011}.

Until recently, the evolutionary progress of cancers have been thought to be gradual and stepwise, where each driver mutation is acquired over many years, or even decades \citep{Stephens2011}.
There are now evidence that show otherwise, whereby a single catastrophic event in the chromosome structure occurs, resulting in major structural rearrangements such as inversions, deletions, duplications and translocations \citep{Leibowitz2015,Stephens2011}.
% This rare event that is estimated to occur in 2$\sim$5\% of all cancers was termed \gls{chromothripsis}, and is thought to be caused by the physical separation of parts of chromosome by a structure called micronuclei \citep{Leibowitz2015,Stephens2011}.
Owing to its nature of being a one-off chromosomal disruption, chromothripsis (usually) cause disruptions in tumour suppressor genes rather than duplications of oncogenes \citep{Leibowitz2015}.

In addition to this, areas of the chromosome where major structural rearrangements have occured can be prone to a regional hypermutation, termed \gls{kataegis} \citep{Leibowitz2015,Nik-Zainal2012}.
Although the full mechanism in which \gls{kataegis} occur is not yet defined, \gls{aid}/\acrshort{apobec} family proteins, a family of proteins involved in gene editing, may play a major role in \gls{kataegis} \citep{Leibowitz2015,Nik-Zainal2012}.
% By deaminating cytidine to uracil in the \gls{ssdna} that may occur at the sites of double strand break, \gls{aid}/\acrshort{apobec} family proteins introduces hypermutated regions in the genome \citep{Leibowitz2015,Nik-Zainal2012}.
It is evident that both \gls{chromothripsis} and \gls{kataegis} contribute to the genomic aberrations and mutations that ultimately encourage the acquisition of many of the tumorigenic phenotypes.

\paragraph{Tumour-promoting inflammation}

\noindent
Inflammation and immune response, though it may assist in the control of diseases and injuries, have been implicated in the progression of tumour and their acquisition of many of the cancer hallmarks \citep{Hanahan2011}.
% There are two pathways of inflammation that contribute to the cancer survival and progression \citep{Mantovani2008}.
% The intrinsic pathway is caused by oncogenic characteristics of the cancer cells, where these cells produce inflammatory signals themselves and promote advantageous characteristics for the cancer cells, such as increased proliferation and survival \citep{Mantovani2008}.
% On the other hand, the external pathway is caused by infectious or inflammatory conditions in the tumour microenvironment that surround the tumour and assist in the growth of the tumour \citep{Mantovani2008}.
% There are two key transcription factors that connect these two pathways of inflammation that ultimately leads to the prommotion  of many, if not all, of the hallmarks discussed so far: \gls{nfkb} and \gls{stat3} \citep{Mantovani2008}.
There are two key transcription factors of inflammation that ultimately leads to the prommotion  of many, if not all, of the hallmarks discussed so far: \gls{nfkb} and \gls{stat3} \citep{Mantovani2008}.

The study by \citep{Pikarsky2004} showed that \gls{nfkb} was an essential component for the progression of hepatocellular carcinoma, providing the first evidence as to how inflammation may affect the tumour biology.
\gls{nfkb} regulates the expression of genes that encode the inflammatory cytokines, angiogenic factors, and many others, including anti-apoptotic genes (through the activation of \gls{stat3}) \citep{Elinav2013,Mantovani2008}.
% \gls{stat3} is a transcription factor that is activated through variety of receptors and pathways, including the \gls{il6} receptor, \glspl{gpcr} and \glspl{tlr} \citep{Yu2007,Yu2014}.
\gls{stat3} is a transcription factor that is activated through variety of receptors and pathways \citep{Yu2007,Yu2014}.
By increasing the transcription of anti-apoptotic proteins such as \gls{bcl2}, \gls{stat3} promotes tumour survival and proliferation \citep{Yu2007}.
Furthermore, \gls{stat3} is able to increase the expression of inflammatory cytokines (such as \gls{il6} and \acrshort{il10}) and \acrshort{vegf}, which provides a positive feedback mechanism for further inflammation (and therefore survival and proliferation signals) and advancement of angiogenesis, respectively \citep{Yu2007}.

% Though helpful at times, inflammation induced by the immune system must be taken care of and controlled in order to prevent rapid tumour development and progression.

\paragraph{Deregulating cellular energetics}

\noindent
In order to grow and proliferate, cancer cells must adjust their energy metabolism to match their energy requirement \citet{Hanahan2011}.
% There are two ways in which a cell can produce energy: anaerobic respiration, where the cell uses glycolysis to rapidly generate \gls{atp} and lactate; and aerobic respiration where, in the presence of oxygen, the cell utilises the electron transport chain and \gls{atp} synthase in the mitochondria to efficiently produce \gls{atp}.
% However, even with abundant oxygen in the cell, \citet{Wardburg1956} found that cancer cells utilised the glycolytic pathway instead of the more efficient aerobic respiration, which was later termed ``aerobic glycolysis'' \citep{Hanahan2011}.
Even with abundant oxygen in the cell, \citet{Wardburg1956} found that cancer cells utilised the glycolytic pathway instead of the more efficient aerobic respiration, which was later termed ``aerobic glycolysis'' \citep{Hanahan2011}.
It seems counterintuitive to limit its full potential of creating \gls{atp} by utilising glycolysis over electron transport chain, as the latter produces almost 18 times more \gls{atp} \citep{Hanahan2011, VanderHeiden2009}.
However, it is hypothesised  that this apprent inefficiency helps the cell to generate and direct the molecules required for biosynthetic pathways during cell growth and division \citep{Cairns2011,VanderHeiden2009}.
By deliberately relying purely on glycolysis (known as the ``Wardberg effect'') and slowing the conversion of \gls{pep} to pyruvate, the cell is able to shunt the carbon molecules into reaction pathways that produce constituents used to synthesise macromolecules, such as \acrshort{dna} and lipids \citep{Cairns2011,VanderHeiden2009}.

The fact that the deregulation of energy metabolism is mediated somewhat by the genes involved in many of the other hallmarks (such as p53 and Myc) suggests that the disruption in energy metabolism may be a phenotype caused in conjunction with those core hallmarks \citep{Hanahan2011}.

\paragraph{Avoiding immune destruction}

\noindent
Our immune system is constantly monitoring and maintaining normal cell biology within our body, preventing the emergence of cancer \citep{Hanahan2011}.
With that said, cancers are frequently observed in clinics, pointing to the fact that the observed cancer cells must have evaded the diligent monitoring by the immune system \citep{Hanahan2011}.

Immunoediting is a process where the immune system successfully eliminates the highly immunogenic cancer cells during surveillance, but fails to eliminate the weakly immunogenic cells and therefore poses a selective advantage on these cells \citep{Hanahan2011,Teng2008}.
In fact, there has been reports where cancer cells in immunodeficient mice were not able to initiate tumour growth in the immunocompetent mice, but the cancer cells from the immunocompetent mice were able to do so in immunodeficient mice \citep{Hanahan2011}.
One possible reason for this was that the cancer cells from the immunodeficient mice were not immunoedited to the extent in which the cells from immunocompetent mice were, and thus unable to evade the immune system as efficiently as those cells in the immunocompetent mice \citep{Hanahan2011}.
Though more experiments must be carried out to confirm its definitive role and mechanisms in which it contributes to cancer biology, immune system evasion is in no doubt a significant component of tumour progression. \\

\noindent
By utilising many, if not all, of these major cancer hallmarks, cancer cells are able to initiate growth, proliferate, invade, metastasise and eventually kill the host if untreated.
There is an evident need for an efficient way to monitor and treat cancer patients, and these Hallmarks of Cancers will prove to be the best target for effective treatments of cancer in the future.

\subsection{Causes and risk factors of cancer}
\label{sub:causes_and_risk_factors_of_cancer}

There are many causes and risk factors associated with the development and progression of cancer.
Like obesity, cancers are dependent on both the genetic and environmental factors.

\subsubsection{Genetic mutations}
\label{ssub:Genetic mutations}

% It is quite obvious from the previous section (\cref{sub:mechanisms_of_cancer_development}) that there are a lot of things that can go wrong in the cell.
Any one of the ten Hallmarks of Cancers has the potential to initiate cancer growth and progression; every hallmark has their own complex mechanistic pathways; every pathway has many proteins associated with its function or role in the cell; and, though their impacts may differ from protein to protein, mutations in any one of the proteins involved in a pathway has the potential to cause cancer.

Oncogenes are genes that promote or ``accelerate'' the process of tumorigenesis and/or tumour progression, whereas tumour suppressor genes prevent or ``decelerate'' those effects \citep{Vogelstein2004a}.
Cancers usually mutate oncogenes in such a way that they become constitutively activated and mutate the tumour suppressor genes so that these genes are inactivated and/or have reduced expression \citep{Vogelstein2004a}.
Another thing to note for these genes is that oncogenes usually only require a ``one-hit'' mutation that affects a single allele, whereas tumour suppressor genes generally require mutations in both maternal and paternal alleles to have any effect \citep{Stratton2009,Vogelstein2004a}.

Stability genes, also known as caretakers, are genes that are involved in, for example,  \gls{mmr}, \gls{ner} and \gls{ber}; in other words, genes that are involved in \acrshort{dna} repair mechanisms \citep{Vogelstein2004a}.
These caretaker genes help prevent the cell from developing new mutations during \acrshort{dna} replication and mitosis.
However, only the oncogenes or the tumour suppressor genes are able to directly affect the biology of the cell, and therefore mutations in stability genes tend to affect the mutation rates of the other classes of genes, ultimately leading to tumorigenesis \citep{Vogelstein2004a}.

There are two types of mutations that occurs in an individual: germline mutation and somatic mutation.
Germline mutations are the mutations that are passed on from an individual's parents, and somatic mutations are the mutations acquired by the cell over the course of the individual's lifetime.

Unfortunately, there are germline mutations that make the individual more prediposed to cancer \citep{Vogelstein2004a}.
These individuals have a mutation in one or more of the classes of cancer-related genes, which makes them more likely to develop a tumour than individuals without the mutation \citep{Vogelstein2004a}.
In contrast, somatic mutations are acquired during an individual's lifetime and is not passed down to their children.
Somatic mutations can either be a driver or a passenger mutation, where a driver mutation helps tumorigenesis and its subsequent progression (in other words, mutations in oncogene, tumour suppressor, or stability gene), and passenger mutation have neither a selective advantage nor contribute to tumour progression \citep{Stratton2009}.

Since cancers require multiple driver mutations, somatic mutations are an essential part of tumour development.
Perhaps that is one of the reasons why, in some cancers, germline mutation in the stability genes cause greater susceptibility to cancers than mutations in oncogene or tumour suppressor gene \citep{Vogelstein2004a}.

\subsubsection{Environmental factors}
\label{ssub:environmental_factors}

% For obesity, since there is a clear evidence of the cause (excessive eating and lack of exercise) and a well defined mechanism of how the disease is developed (excess calories are converted into fat), it is relatively easy to speculate the effect of the environment on the disease (for example, abundance of high calory food).
Cancers are developed through many cellular pathways (Hallmarks of Cancers) which are not fully uncovered yet, with either too many or very few causes associated with those pathways.
Therefore it is difficult to assess how, rather than whether, a given environmental factor causes specific cancer.
With that said, several environmental factors have a strong evidence of causing cancer.
% , though the complete mechanism of action for some of these must be elucidated in the coming years.
% Out of these diet and obesity will be covered as an example.

Smoking is a known risk factor for many cancers; the obvious one being lung cancer \citep{Gandini2008,Hecht1999}.
For current smokers, the relative risk is greater than 1.5 in many cancers, including liver, cervix and kidney, and it can rise up to 8.96 in lung cancer \citep{Gandini2008}.
% Even if smokers stop smoking, the relative risk does not decrease until the first 5 years have passed and will never return to the level of non-smokers \citep{Hecht1999}.
% In tobacco smoke, there are literally hundreds and thousands of chemical compounds, of which about 20 compounds have been shown to have carcinogenic effects \citep{Hecht1999}.
% Among these, \glspl{pah} and \gls{nnk} have been shown to cause lung cancer in mice when administered systemically \citep{Hecht1999}.
Tobacco smoke contain many carcinogens which are processed first by \gls{cyp} enzymes (termed meta\-bolic activation), and the product can either be processed further to be detoxified completely and cause no harm to the cell, or alternatively, the product can covalently bind to the \acrshort{dna}, forming \acrshort{dna} adducts \citep{Hecht1999}.
Unless these adducts are resolved correctly by the \acrshort{dna} repair mechanisms, it can result in a permanent mutation within the genome and lead to oncogenesis \citep{Hecht1999}.
% In addition to this, cigarettes contain free radicals and chemicals that cause tremendous oxidative stress and damage to the cells \citep{Hecht1999}.
% These radicals cause \acrshort{dna} nicks and damages the \acrshort{dna}, again requring the cell's \acrshort{dna} repair mechanisms to remove the damage so it does not become permanent.

% Another important environmental factor to consider is \gls{uv} radiation.
Exposure to \gls{uv} radiation has been significantly associated with the increased risks of skin cancers \citep{Armstrong2001,Gallagher2006}.
% The risks of developing any skin cancers can vary between race (Caucasian people are more likely to get skin cancers), and potentially the age when they were exposed to high amounts of \gls{uv} radiation \citep{Armstrong2001,Gallagher2006}.
Similar to smoking, the mechanism of how \gls{uv} radiation may cause  skin cancers is related to the \acrshort{dna} repair system.
\gls{uv} radiation causes the formation of a dimer in the adjacent pyrimidines, resulting in a covalent bond formation between these nucleotides \citep{Friedberg2003,Hoeijmakers2001}.
These dimers introduce bulky structures in the \acrshort{dna} that prevent the \acrshort{dna} replication and transcription enzymes from functioning properly, and it may cause lasting mutations if not repaired \citep{Friedberg2003,Hoeijmakers2001}.
These damages caused by \gls{uv} radiation can be relieved by the \gls{ner} system or the photoreactivation process, where a photoreactivating enzyme uses light to monomerize the dimeric pyrimidines \citep{Friedberg2003}.

Diet and obesity are known risk factors for many types of cancers as well \citep{Ames1995a,Calle2004}.
% Though more will be covered in details in \cref{sec:obesity_and_cancer}, the effect of obesity on cancers will be briefly mentioned here.
Epidemiological evidence of obesity being a risk factor for many cancers has been well established, yet the mechanisms of its contribution to cancer is still  speculative \citep{Calle2003,Kelesidis2006a}.
Although there are a lot of possible mechcanisms, there is growing evidence of the role of chronic inflammation caused by increased fat mass in the body \citep{Kelesidis2006a,Lumeng2011,Hernandez2013}.
Free fatty acids from excess adipose tissue and/or diet can be recognised by \gls{tlr} (specifically \gls{tlr}4), which can then activate \gls{nfkb}, leading to the activation of one of the Hallmarks of Cancer -- tumour-promoting inflammation \citep{Lumeng2011}.
% The nutritional content of the food that people consume is also important, as it can both directly and indirectly influence the likelihood of tumorigenesis, as well as its progression \citep{Ames1995a}.
% Many fruits and vegetables contain antioxidants as well as essential viatmins and minerals.
% Antioxidants are able to neutralise the effect of \gls{ros},  and many viatmins have been associated with reduction of many types of cancers \citep{Ames1995a}.
% In fact, ascorbic acid (vitamin C) is able to prevent the lipid peroxidation of lipid plasma caused by tobacco smokes \textit{in vitro}, which could be why smokers, compared to non-smokers, have decreased levels of ascorbic acid \citep{Hecht1999}.
% By maintaining healthy diet, it may be possible to reduce the risks of many cancers and obesity, as well as the comorbidities associated with them.

% As one can imagine, \acrshort{dna} damages caused by smoking and/or \gls{uv} radiation have the potential to damage the genome and initiate cancer if wrong genes are mutated, such as p53 tumour suppressor gene.
% Healthy diet is also crucial for lowering the risk of cancers through beneficial vitamins and antioxidants, and prevent obesity as well as chronic inflammation that accompanies with it.
There are many other environmental factors associated with cancers, and limiting the exposure to detrimental environmental factors is crucial for the prevention of any type of cancers, as well as the more specific type of cancers associated with these factors, such as lung and skin cancers.

\subsection{Treatment of cancer}
\label{sub:treatment_of_cancer}

Traditionally, the treatment of cancer has been surgery, radiation therapy and chemotherapy.
Though surgery is an effective method to remove the tumour from the patient, there is always a risk of metastasis and recurrence, which is why combination of surgery with other treatments like chemotherapy is usually considered.
Earlier chemotherapeutic drugs and radiation therapy had limited effect on tumours due to their resistance against these treatments \citep{Wilhelm2006}.
However, better understanding of the pathways and mechanisms used by the tumour cells have shed  light on the development of drugs that are more effective against tumours.
RAF inhibitors are among one of the many chemotherapeutic drugs.

The RAS-RAF-MEK-ERK signalling pathway is  involved in cell survival, growth and proliferation \citep{Samatar2014,Wilhelm2006}.
The members of this pathway are often mutated in many types of cancer (approximately a third of cancers contain a mutation in the RAS oncogene), causing an overactivation of this pathway \citep{Samatar2014}.
Development of drugs that target  the downstream RAF and its mutant forms (such as BRAF$^{V600E}$) have been successful, and many RAF inhibitor drugs such as sorafenib and vemurafenib have been approved for clinical use \citep{Samatar2014,Wilhelm2006}.
Even though the signalling pathways are well understood and the quality of the drugs that are developed based on these knowledge is improving, there is always the possibility of tumour to acquire resistance to those drugs \citep{Samatar2014}.

In fact, there has been evidence of pathway cross-talks between RAS-ERK pathway and \gls{pi3k}/Akt pathway, which may help the tumour cells achieve resistance against those drugs \citep{Moelling2002,Zimmermann1999a}.
A study in a \textit{Drosophila} model showed that the inhibition of just a single pathway does not necessarily stop the effect of the pathway, as the signal can flow through alternate pathways \citep{Dar2012}.
Pathway cross-talk and increased flux through alternate pathways could be one of the ways in which tumour cells acquire resistance to drugs.
However, the effectiveness of targeting multiple pathways in humans is yet to be confirmed and further investigations are required to solidify the mode of action and its relevance to tumour biology.

Further understanding of the tumour biology, its mechanism of resistance to drugs, and the effect of targeting multiple pathways (or targeting effective pathways for a patient) will definitely become important in the near future, both for novel drug development and better clinical treatment plan for individual patients.

\section{Obesity and cancer}
\label{sec:obesity_and_cancer}

As mentioned briefly in \cref{sub:causes_and_risk_factors_of_cancer}, obesity is considered a major risk factor for many types of cancer.
The fact that the prevalence of obesity is extremely high, it is important to establish the link between the apparent risks associated with cancer.

\subsection{Cancer risks associated with obesity}
\label{sub:cancer_risks_associated_with_obesity}

Although the mechanisms in which obesity causes cancer is still under debate, the association between obesity and cancer has long been established.
Perhaps the most convincing evidence of the association between obesity and cancer was first presented by \citet{Calle2003}.
In this study, the link between \gls{bmi} and variety of cancers were investigated in 900,000 adults between 1982 and 1998 \citep{Calle2003}.
The results from this study showed that morbidly obese men and women (\gls{bmi} \textgreater{} 40) had 52\% and 62\% higher chance of dying from all cancers compared to normal weight adults (18.5 \textgreater{} \gls{bmi} \textgreater{} 25), respectively \citep{Calle2003}.
\gls{bmi} was also associated with higher rates of death from esophageal, colorectal, liver, gallbladder, pancreas, kidney, and some sex-specific cancers \citep{Calle2003}.
Furthermore, the death rates had a positive linear correlation with increasing \gls{bmi} in all cancers \citep{Calle2003}.

In a more recent meta-analysis study, \citet{Renehan2008} showed that an increase in \gls{bmi} by 5 kg/m$^2$ significantly increased the risks of esophageal, thyroid, colon and renal cancers in men, and endometrial, gallbladder, esophageal and renal cancers in women.
The summary of the risks associated with various cancers from this study are shown in \cref{tab:renehan_cancer_risks}.
This meant that the association between \gls{bmi} and cancers were sex-specific, applicable to wide range of cancer types, and generally consistent across different geographic populations \citep{Renehan2008,Roberts2010}.
It is undeniable that obesity is a major risk factor for many types of cancer and it may even be the cause for some of these cancers.

% Taken together, it is undeniable that obesity is a major risk factor for many types of cancers, and it may even be the cause for some of these cancers.

\begin{table}[htb]
	\centering
	\begin{threeparttable}
		\caption{Summary of the risk ratios\tnote{1} associated with each type of cancers from the meta-analysis study by \citet{Renehan2008}\tnote{2}.}
		\label{tab:renehan_cancer_risks}
		\begin{tabular}{lcc}
													   & Men               & Women\\
			\hline
			\rule{0pt}{2.25ex}Colon                 & 1.24 (1.20, 1.28) & 1.09 (1.05, 1.13)\\
			Rectum                                  & 1.09 (1.06, 1.12) & NA\tnote{3}\\
			Gallbladder                             & NA                & 1.59 (1.02, 2.47)\\
			Leukemia                                & 1.08 (1.02, 1.14) & 1.17 (1.04, 1.32)\\
			Malignant melanoma                      & 1.17 (1.05, 1.30) & NA\\
			Multiple myeloma                        & 1.11 (1.05, 1.18) & 1.11 (1.07, 1.15)\\
			Non-Hodgkin lymphoma                    & 1.06 (1.03, 1.09) & 1.07 (1.00, 1.14)\\
			Esophageal adenocarcinoma               & 1.52 (1.33, 1.74) & 1.51 (1.31, 1.74)\\
			Pancreatic                              & NA                & 1.12 (1.02, 1.22)\\
			Renal                                   & 1.24 (1.15, 1.34) & 1.34 (1.25, 1.43)\\
			Thyroid                                 & 1.33 (1.04, 1.70) & 1.14 (1.06, 1.23)\\
			Prostate                                & 1.03 (1.00, 1.09) & NA\\
			Post-menopausal breast                  & NA                & 1.12 (1.08, 1.16)\\
			Endometrial (\textless{}27 kg/m$^2$)    & NA                & 1.221 (1.084, 1.376)\\
			Endometrial (\textgreater{}27 kg/m$^2$) & NA                & 1.729 (1.598, 1.872)\\
			\hline
			\hline
		\end{tabular}
		\begin{tablenotes}
			\begin{footnotesize}
				\item [1] All risk ratios are per increase in 5 kg/m$^2$ \gls{bmi}; 95\% confidence intervals in brackets.
				\item [2] Table adapted from \citet{Roberts2010} paper, where the original data in the table was taken from \citet{Renehan2008} study.
				\item [3] Not applicable.
			\end{footnotesize}
		\end{tablenotes}
	\end{threeparttable}
\end{table}

\subsection{Mechanisms of cancer progression in obese patient}
\label{sub:mechanisms_of_cancer_progression_in_obese_patient}

The exact mechanisms in which obesity causes cancer have still not been fully elucidated yet.
There are several hypotheses that link obesity with cancer: disruption of various hormone levels, chronic and sustained inflammation, and activation of the insulin/\gls{igf} pathway.

% Previously thought to be a group of cells where the body could store and manage excess fats, it has recently become apparent that adipose tissue is also a metabolically active organ that has a significant role in the endocrine system \citep{Roberts2010}.
Adipose tissue has long been thought to be a group of cells to store excess fats, but it has recently become apparent that adipose tissue is also a metabolically active endocrine organ \citep{Roberts2010}.
Increased adiposity is known to influence the bioavailability of endogenous sex hormones, including estrogens, androgens and progesterone \citep{Calle2004}.
\Glspl{adipocyte} synthesise enzymes, such as aromatase, that convert precursor molecules into different forms of androgens and estrogens \citep{Calle2004}.
Many studies have confirmed the mitogenic effect of sex hormones on certain cell types, mainly breast and endometrial cells \citep{Roberts2010}.

On top of the effects that \glspl{adipocyte} have on circulating concentration of sex hormones, they also produce adipose tissue-specific hormones called adipokines \citep{Roberts2010}.
Of the many adipokines identified so far, leptin and adipo\-nectin have been studied the most in terms of cancer development \citep{Renehan2006,Roberts2010}.
Leptin is known to activate not only the satiety signalling pathway, but also the \gls{jak}/\gls{stat} and \gls{pi3k}/Akt signalling pathways \citep{Garofalo2006,Renehan2006}.
Activation of these pathways lead to the activation of target genes that are involved in cell proliferation and cell survival, as well as pro-angiogenic factors \citep{Garofalo2006}.

Adiponectin is a hormone that is synthesised and secreted only by the \glspl{adipocyte} \citep{Kelesidis2006a}.
Unlike leptin, where the concentration of leptin is higher in obese people, adiponectin levels are lower in obese people compared to the normal weight people \citep{Kelesidis2006a,Renehan2006}.
Adiponectin has anti-angiogenic and anti-proliferative effect on the cells, and is secreted by mature adipocytes but not by the premature pre-adipocytes \citep{Gilbert2013}.
During chronic inflammatory state in obese patients, pro-inflammatory proteins such as \gls{tgfb} and \gls{ifny} can block the maturation of the pre-adipocytes, thus decreasing the level of adiponectin and the accompanying beneficial effects \citep{Gilbert2013}.

Alternatively, or possibly in conjunction with hormone disruption, inflammation caused by obesity may be associated with tumour-promoting environment.
\Gls{tlr}4 is a type of \gls{prr} and, as mentioned earlier, is able to recognise circulating free fatty acids and trigger immune response via \gls{nfkb} pathway \citep{Lumeng2011}.
In addition to \gls{nfkb} pathway, recent evidence have shown that \gls{tlr}4 could also activate the \gls{jak}/\gls{stat3} pathway \citep{Yu2014}.
Activation of these pathways will further enhance the development of the tumour-promoting microenvironment caused by these unwanted inflammatory responses of obesity.
Furthermore, since it is likely that obese people have less adiponectin concentration, obesity-induced inflammation may affect these patients more than those with normal weight.

Finally, insulin/\gls{igf} pathway may have a major role in encouraging tumorigenesis in obese patients.
% \Gls{igf} system involves two ligands (\gls{igf}-I and \gls{igf}-II), its respective receptors (\gls{igf}-IR and \gls{igf}-IIR), and six binding proteins (\gls{igfbp} 1 to 6) \citep{Roberts2010}.
\gls{igf} signalling results in the activation of \gls{pi3k}/Akt and \gls{erk} pathways, resulting in cell proliferation and reduced apoptosis \citep{Roberts2010}.
Though insulin is also capable of activating these pathways, it is thought that the mitogenic effects are controlled mainly through the \gls{igf} receptors \citep{Roberts2010}.
Since insulin reduces the level of \gls{igfbp}-3, the main circulating \gls{igfbp}, hyperinsulinemia has been hypothesised to promote tumorigenesis through the mitogenic effect of insulin and free \glspl{igf} \citep{Giovannucci1995,Mckeown1994,Roberts2010}.
In fact, many studies have consistently associated the increased concentration of free \gls{igf}-I with obesity and increased risk of some cancers; however the results for \gls{igfbp} have been inconsistent \citep{Basen2011}.

It is evident that there are many possible mechanisms in which obesity contributes to the progression of cancer.
Furthermore, these mechanisms are related to one another and are part of a larger, more complex physiological network \citep{Renehan2006}.
Due to this complexity, it is difficult to confidently explain how obesity contributes to tumour development.

\section{Genetic signatures}
\label{sec:genetic_signatures}

One of the difficulties associated with the treatment of cancers is the identification of the underlying mechanisms or hallmarks that drive the cancer progression.
As one can imagine, it is difficult for clinicians to treat a patient if they do not know what is causing the disease.
The reason why it is so difficult to identify the underlying biological mechanisms is that there are very little reliable biological markers (or ``genetic signatures'') that signify the presence or absence of these pathways.
With that said, significant improvements have been made in recent years owing to the advancement in sequencing technologies and have allowed researchers to study the genetic signatures that represent the hallmarks in a much greater detail.

\subsection{Microarray and Next Generation Sequencing technologies}
\label{sub:microarray_and_next_generation_sequencing_technologies}

\subsubsection{Microarray technology}
\label{ssub:microarray_technology}

Microarray technology was first developed by \citet{Schena1995} and has been used extensively to study gene expression patterns and \gls{snp} identification for over two decades.
In short, microarray technology uses a glass slide with hundreds of thousands of short \acrshort{dna} sequences attached to it, where fluorescently labelled \glspl{cdna} taken from the sample \glspl{mrna} are hybridised to these immobilised \acrshort{dna} sequences, and the fluorscent probes are detected to measure the gene expression \citep{Schena1995,Schulze2001}.
% Since only the exactly matching sequences remain on the microarray slide, researchers are able to identify and quantify the gene expression level of the sample \citep{Schulze2001}.

One disadvantage of microarray technology is that it requires previous knowledge of the genes and \glspl{snp} of the samples being investigated \citep{Hurd2009}.
This is due to the fact that the detection of sequences rely on the hybridisation of the sample sequences with the microarray sequences and therefore the sequences under investigation must be known to attach the correct sequences to the microarray.
Another disadvantage of microarrays is the potentially erroneous detection of the sample sequences, both in terms of its sequence and its expression level \citep{Hurd2009}.
Again, because the microarray slides require pre-defined sequences to be attached to it, there is always a possibility of hybridisation of similar but different sequence to the slide.
Even if the sequences correctly hybridise to the slide, it may be difficult to measure the exact amount of some sequences in the samples, since the amount of sequences in the sample is measured with the relative strength of the fluorescence of the dyes used \citep{Hurd2009}.
Therefore, it is difficult to detect the exact level of expressions of sequences that are either very rare or highly abundant in the sample \citep{Hurd2009}.

\subsubsection{Next generation sequencing technology}
\label{ssub:next_generation_sequencing_technology}

\Gls{ngs} technology refers to the high throughput sequencing techniques that allow sequencing of large quantity of genetic material \citep{Metzker2010}.
\gls{ngs} technology has been developed and improved in the last decade at a revolutionary pace, making it possible for researchers to study gene expressions and variations in days to weeks, where the same experiment would have taken years to complete before the technology was developed.

Every \gls{ngs} technology uses the same principle process to sequence \acrshort{dna} samples.
First, \acrshort{dna} samples are fragmented into smaller sizes and universal primers are attached to the ends of the fragments.
% These fragments are then amplified into clusters to make the signals stronger for detection (although this amplification step vary between different techniques) \citep{Metzker2010}.
These fragments are then amplified into clusters to make the signals stronger for detection \citep{Metzker2010}.
After the amplification step, the clusters of \acrshort{dna} fragments are sequenced and ``imaged'' using fluorescently labelled \acrshort{dna} bases, where the incorporation of different bases are detected by its unique fluorescence \citep{Metzker2010}.

\gls{ngs} technology have many advantages over microarray technology.
Firstly, \gls{ngs} does not require any prior knowledge of the sample like in the microarray technology, as all genetic materials are attached to universal primers when they are sequenced; there is no need to pre-define the sequences to be searched and thus allow the detection of novel genes or transcripts \citep{Hurd2009}.
This also prevents the introduction of experimental bias, as the sequences are not altered in any way and no specific sequence is searched in a greater detail over another \citep{Hurd2009}.
Another advantage is that \gls{ngs} produce count data of the genetic material being sequenced instead of relying on the intensity of the fluorescence to detect the presence and amount of certan sequence.
This allows an absolute quantification of the amount of transcripts from the sample, making it possible for a more accurate comparison of the expression levels between samples.
Lastly, many \gls{ngs} techniques are capable of sequencing multiple samples simultaneously with high throughput, generating tremendous amount of genetic data in a single run.

The use of \gls{ngs} technology have significantly improved the workflow of modern genetic studies and its application in variety of research area have made it possible to dig deeper into the causes of complex diseases, such as cancer.
With the cost of sequencing declining, personalised genome sequencing and treatment based on the sequence will be a reality in the near future.

\subsection{Obesity associated genetic signatures}
\label{sub:obesity_associated_genetic_signatures}

There is a clear evidence of obesity being a major risk factor for many cancer types, but what genes or pathways directly contribute to tumorigenesis in patients that are obese is yet to be clarified.
Two studies in particular, one by \citet{Creighton2012} and another by \citet{Fuentes-Mattei2014}, have investigated if there were any difference in genetic composition between obese and normal weight patients, specifically in a group of breast cancer patients.
In both studies, the authors were able to identify a set of genes that were up or downregulated in breast tumours from obese patients.
% For simplicity, these genetic signatures will be referred to as ``obesity associated genetic signatures'' hereafter.

\subsubsection{\citet{Creighton2012} study}
\label{ssub:creighton_study}

\citet{Creighton2012} used 103 breast tumour samples and analysed these samples using microarrays.
Of the 103 samples, 35\% were from normal weight, 28\% were from overweight and 37\% were from obese patients.
It was also noted that African American patients were more likely to be obese than the Caucasian patients.
They compared the gene expression between the tumours from obese patients with the tumours from non-obese patients and identified 799 probe sets (662 unique genes) with p \textless{} 0.01 and with a \gls{fc} \textgreater{} 1.2.
Majority of these genes (725 gene probes) were downregulated and 74 gene probes were upregulated in the samples from obese patients compared with the samples from non-obese patients.

\subsubsection{\citet{Fuentes-Mattei2014} study}
\label{ssub:fuentes_mattei_study}

\citet{Fuentes-Mattei2014} studied 137 breast tumour samples from \gls{er} positive patients and analysed their transcriptome using microarrays.
Of the 137 patients, 94 patients were not obese and 43 patients were.
Comparison of the gene expression of the tumour samples from obese patients with the samples from non-obese patients revealed 112 genes that were statistically significantly dysregulated (62 genes upregulated and 50 genes downregulated).
Furthermore, functional transcriptomic analysis suggested that insulin signalling and inflammation affected human \gls{er}+ breast cancer.

On top of this, \citet{Fuentes-Mattei2014} validated their results using a mouse model, where they compared the expression level in the breast tumours from obese mice with the samples from lean mice.
Results from this animal model provided strong evidence that obesity significantly accelerates tumour growth through Akt/\gls{mtor} signalling pathway.

\subsection{Pathway associated genetic signatures}
\label{sub:pathway_associated_genetic_signatures}

Like obesity associated genetic signatures, there has also been a focus on the identification of pathways and genes that affect tumour growth.
About a decade ago, \citet{Bild2006} demonstrated that a collection of genes that are involved in a specific pathway could be used to determine the status of those pathways in a sample.
These pathway specific genes were identified experimentally by altering the expression of one specific gene that is central to the pathway in a model cell culture system.
For example, all of the genes that have altered its expression in the cells that have \textit{Myc} gene induced were noted down as the genes involved in the Myc pathway.

\subsubsection{\citet{Gatza2010a} study}
\label{ssub:gatza_study}

\citet{Gatza2010a} used 18 pathway signatures to classify human breast tumours into more detailed subtypes.
The 18 pathways used in this study were: Akt, \gls{bcat}, E2F1, \gls{egfr}, \gls{er}, \gls{her2}, \gls{ifna}, \gls{ifny}, Myc, p53, p63, \gls{pi3k}, \gls{pr}, Ras, \gls{stat3}, Src, \gls{tgfb}, and \gls{tnfa} pathways.
This was a significant improvement from their previous works on pathway signatures \citep{Bild2006,Bild2009}.
Using more pathways in their analysis have allowed them to identify certain pathways to be coactivated (or cluster) together, suggesting that some pathways are related to one another and may have relevance in identifying the causes.

One thing to note here is that most of these pathway associated genetic signatures are involved in one or more of the Hallmarks of Cancer.
This means that these pathway associated signatures can be used to provide insights into the underlying biological mechanisms of the samples used in this project.

\section{Aim of the project}
\label{sec:aim}

This research aims to determine whether gene expression signatures exist  that are specific to obesity across multiple cancer types, and to investigate whether there are any common pathways being dysregulated in cancers based on these genetic signatures.
Better understanding of the pathways being dysregulated in cancer cells in obese patients may lead to improved clinical decisions, and contribute towards personalised treatment in the future.

\section{Thesis outline}
\label{sec:thesis_outline}

In the following chapter, the detailed methods of all the analyses carried out for this project is outlined.
The results from these analyses are split into two parts, each presented in separate chapters: first part of the results clarify the relationship between the obesity associated genetic signatures with various cancer types; and the second part of the results consist of the links between the pathway associated genetic signatures and various cancer types, as well as the previously identified obesity associated signatures.
By the end of this thesis, the aim presented here will be answered with clarity.

% TODO: Finished draft copy of introduction -- need to cut down on the contents and link everything together
