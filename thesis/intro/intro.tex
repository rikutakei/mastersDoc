\chapter{Introduction (draft)}
\label{ch:intro}

% TODO: also have a look at the masters talk write up from 2015

\section{Obesity}
\label{sec:obesity}

% General intro to obesity.
% Implications of obesity -- disease associated with it and/or cancer.

\subsection{Prevalence of obesity}
\label{sub:prevalence_of_obesity}



\subsection{Obesity and cancer risk}
\label{sub:obesity_and_cancer_risk}




\section{Cancer}
\label{sec:cancer}

% What is cancer.

\subsection{Hallmarks of Cancer}
\label{subsec:cancerhallmarks}

% TODO: edit the starting sentence so it's not a copy from the beginning
% It is now commonly understood that cancers are caused by dysregulation of various mechanisms or pathways that allow tumour cells to prolilferate, survive and migrate.
% Many of these dysregulated pathways frequently occur in cancer cells, and these characteristic traits, commonly known as the ``Hallmarks of Cancer'', were first described by \citet{Hanahan2000}.

% These hallmarks were initially presented in 2000 to provide a common framework for scientists to better understand cancer biology, and also to direct the focus of future research.
% After a decade of advances in cancer research, the ``Hallmarks of Cancer'' were revisited by Hanahan and Weinberg in 2011, and four more hallmarks were added on top of the six presented in the original paper in 2000 \citep{Hanahan2011}.

% Together, these ``Hallmarks of Cancer'' (fig 1) have provided a foundation of the general biological knowledge of cancer and its characteristics that allowed the researchers to better understand the underlying cancer biology.

% Although much reseach has been conducted on the ``Hallmarks of Cancer'', the molecular processes underlying these hallmarks have still not been fully explained \citep{Hanahan2011}.
% This is due to the fact that cancer cells develop progressively under selective pressures, and are also affected by many environmental factors \citep{Gatza2011}, which makes it difficult to identify causal factors of cancer development.
% Furthermore, these external factors affect how the cancer cells respond to the environment, and may also cause cancer progression and/or drug resistance.

\subsection{Tumour heterogeneity}
\label{sub:tumour_heterogeneity}




\subsection{Relationship between obesity}
\label{subsec:obsbackground}

% Why is it important in obesity context.

% The exact mechanism in which obesity accelerates cancer progression is unknown, but obesity is thought to disrupt the levels of various hormones (such as insulin, leptin, adeponectin, and other adipokines) that eventually activate tumorigenic pathways (reference the overview on obesity and tumour resistance).

\section{Genetic signatures}
\label{sec:genetic_signatures}

\subsection{Microarray technology}
\label{subsec:microarray_technology}

\subsection{Next generation sequencing technology}
\label{sub:next_generation sequencing_technology}



% Due to advances in sequencing technology over the last few years, many researchers have started to focus on bioinformatics based analyses of cancer cells to better understand the biology of cancer.
% Advantages of this strategy are that very specific gene expression and/or mutational signatures for certain cancer subtypes can be identified from the analyses, which the researchers and clinicians can study in much greater depth and details.

% TODO: have a leading sentence that introduces the gene signatures

\section{Obesity associated genetic signatures}
\label{sec:obesity_associated_genetic_signatures}

\subsection{\citet{Creighton2012} study}
\label{sub:creighton_study}



\subsection{\citet{Fuentes-Mattei2014} study}
\label{sub:fuentes_mattei_study}




\section{Pathway associated genetic signatures}
\label{sec:pathway_associated_genetic_signatures}

% Talk about \citet{Gatza2010a}





\section{Aim of the project}
\label{sec:aim}




