\chapter{Introduction (draft)}
\label{ch:intro}

% TODO: also have a look at the masters talk write up from 2015

% This is taken from the committee report (I think it's a good intro to the intro)
% TODO: past tense...??
It is now commonly understood that cancers are caused by dysregulation of various mechanisms or pathways that allow tumour cells to prolilferate, survive and migrate.
There have been many reports on the association between obesity and worsened cancer prognosis, suggesting that there may be distinct changes that underlie tumorigenesis and/or cancer progression that are specific to patients with higher \gls{bmi}.

% TODO: in the aims section, or here?
In brief, this research aims to determine whether gene expression signatures exist that are specific to obesity across multiple cancer types, and to investigate whether there aare any common pathways being dysregulated in cancers based on these genetic signatures.
Better understanding of the pathways being dysregulated in cancer cells in obese patients may lead to improved clinical decisions, and could thus contribute towards personalised treatment in the future.


\section{Obesity}
\label{sec:obesity}

General intro to obesity.
Implications of obesity -- disease associated with it and/or cancer.

\subsection{Prevalence of obesity}
\label{sub:prevalence_of_obesity}



\subsection{Obesity and cancer risk}
\label{sub:obesity_and_cancer_risk}




\section{Cancer}
\label{sec:cancer}

What is cancer.

\subsection{Hallmarks of Cancer}
\label{subsec:cancerhallmarks}

% TODO: edit the starting sentence so it's not a copy from the beginning
It is now commonly understood that cancers are caused by dysregulation of various mechanisms or pathways that allow tumour cells to prolilferate, survive and migrate.
Many of these dysregulated pathways frequently occur in cancer cells, and these characteristic traits, commonly known as the ``Hallmarks of Cancer'', were first described by \citet{Hanahan2000}.

These hallmarks were initially presented in 2000 to provide a common framework for scientists to better understand cancer biology, and also to direct the focus of future research.
After a decade of advances in cancer research, the ``Hallmarks of Cancer'' were revisited by Hanahan and Weinberg in 2011, and four more hallmarks were added on top of the six presented in the original paper in 2000 \citep{Hanahan2011}.

Together, these ``Hallmarks of Cancer'' (fig 1) have provided a foundation of the general biological knowledge of cancer and its characteristics that allowed the researchers to better understand the underlying cancer biology.

Although much reseach has been conducted on the ``Hallmarks of Cancer'', the molecular processes underlying these hallmarks have still not been fully explained \citep{Hanahan2011}.
This is due to the fact that cancer cells develop progressively under selective pressures, and are also affected by many environmental factors \citep{Gatza2011}, which makes it difficult to identify causal factors of cancer development.
Furthermore, these external factors affect how the cancer cells respond to the environment, and may also cause cancer progression and/or drug resistance.

\subsection{Tumour heterogeneity}
\label{sub:tumour_heterogeneity}




\subsection{Relationship between obesity}
\label{subsec:obsbackground}

Why is it important in obesity context.

The exact mechanism in which obesity accelerates cancer progression is unknown, but obesity is thought to disrupt the levels of various hormones (such as insulin, leptin, adeponectin, and other adipokines) that eventually activate tumorigenic pathways (reference the overview on obesity and tumour resistance).

\section{Genetic signatures}
\label{sec:genetic_signatures}

Due to advances in sequencing technology over the last few years, many researchers have started to focus on bioinformatics based analyses of cancer cells to better understand the biology of cancer.
Advantages of this strategy are that very specific gene expression and/or mutational signatures for certain cancer subtypes can be identified from the analyses, which the researchers and clinicians can study in much greater depth and details.

% TODO: have a leading sentence that introduces the gene signatures

\subsection{Obesity associated genetic signatures}
\label{subsec:obesity_associated_genetic_signatures}

\subsubsection{ \textbf{\citet{Creighton2012} study}}
\label{ssub:creighton_study}



\subsubsection{ \textbf{\citet{Fuentes-Mattei2014} study}}
\label{ssub:fuentes_mattei_study}




\subsection{Pathway associated genetic signatures}
\label{sub:pathway_associated_genetic_signatures}

Talk about \citet{Gatza2010a}





\section{Aim of the project}
\label{sec:aim}




\newpage

CONSISTENCY OF VERB TENSE helps ensure smooth expression in your writing. The practice of the discipline for which you write typically determines which verb tenses to use in various parts of a scientific document. In general, however, the following guidelines may help you know when to use past and present tense. If you have questions about tense or other writing concerns specific to your discipline, check with your adviser.\\

USE PAST TENSE. . .\\

To describe your methodology and report your results.\\

At the time you are writing your report, thesis, dissertation or article, you have already completed your study, so you should use past tense in your methodology section to record what you did, and in your results section to report what you found.\\

       We hypothesized that adults would remember more items than children.\\

       We extracted tannins from the leaves by bringing them to a boil in 50% methanol.\\

       In experiment 2, response varied.\\

When referring to the work of previous researchers.\\

When citing previous research in your article, use past tense. Whatever a previous researcher said, did or wrote happened at some specific, definite time in the past and is not still being done. Results that were relevant only in the past or to a particular study and have not yet been generally accepted as fact also should be expressed in past tense:\\

Smith (2008) reported that adult respondents in his study remembered 30 percent more than children. (Smith's study was completed in the past and his finding was specific to that particular study.)\\

Previous research showed that children confuse the source of their memories more often than adults (Lindsey et al., 1991). (The research was conducted in the past, but the finding is now a widely accepted fact.)\\

To describe a fact, law or finding that is no longer considered valid and relevant.\\

Nineteenth-century physicians held that women got migraines because they were "the weaker sex," but current research shows that the causes of migraine are unrelated to gender. (Note the shift here from past tense [discredited belief] to present [current belief].)\\

USE PRESENT TENSE. . .\\

To express findings that continue to be true.\\

Use present tense to express general truths or facts or conclusions supported by research results that are unlikely to change - in other words, something that is believed to be always true:\\

       Genetic information is encoded in the sequence of nucleotides on DNA.\\

       Galileo asserted that the earth revolves the sun.(The asserting took place in the past, but the\\
       earth is still revolving around the sun. Note also that no source citation is needed here since it is a\\
       widely known and well-accepted fact that Galileo made this assertion.)\\

       Sexual dimorphism in body size is common among butterflies  (Singer1982).(Note how this\\
       statement differs from one in which you refer to the researcher's work in the sentence: "Singer\\
       (1982) stated that sexual dimorphism in body size is common among butterflies." Here you use past\\
       tense to indicate what Singer reported, but present tense to indicate a research result that is\\
       unlikely to change.)\\

       We chose Vietnam for this study because it has a long coastline. (Use past tense to indicate what\\
       you did [chose Vietnam], but present tense to indicate you assume that the length of Vietnam's\\
       coastline is unlikely to change.)\\

       We used cornmeal to feed the fingerlings because it provides high nutritional content at a\\
       relatively low cost. (Past tense reflects what you did [used cornmeal], but present tense indicates\\
       that neither the nutritional content nor the cost of corn meal is likely to change.)\\

To refer to the article, thesis or dissertation itself.\\

Use the present tense in reference to the thesis or dissertation itself and what it contains, shows, etc. For example:\\

       Table 3 shows that the main cause of weight increase was nutritional value of the feed. (Table 3\\
       will always show this; it is now a fact that is unlikely to change, and will be true whenever anyone\\
       reads this sentence, so use present tense.)\\

To discuss your findings and present your conclusions. Also use present tense to discuss your results and their implications.\\

       Weight increased as the nutritional value of feed increased. These results suggest that feeds\\
       higher in nutritional value contribute to greater weight gain in livestock. (Use past tense to\\
       indicate what you found [weight increased], but use present tense to suggest what the result\\
       implies.)\\

