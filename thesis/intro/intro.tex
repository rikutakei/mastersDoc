\chapter{Introduction (draft)}
\label{ch:intro}

% TODO: also have a look at the masters talk write up from 2015

\section{Obesity}
\label{sec:obesity}

% General intro to obesity.
% Implications of obesity -- disease associated with it and/or cancer.
Obesity has been one of the major global problems for more than a decade, associated with many noncommunicable diseases such as diabetes, cardiovascular diseases and certain types of cancers \citep{WHO2014}.
In fact, the risk of comorbidities increases with the increase in \gls{bmi}, where the risk becomes severe as the \gls{bmi} level approaches the obese category \citep{WHO2000}.

The number of overweight and obese population, in both adults and children, have risen in every country of the world, and the trend continues as we speak.
Estimated to account for 3.4 million deaths per year and 93.6 million \glspl{daly} in 2010, obesity is a serious disease that continues to grow in our society \citep{Lim2012}.

\subsection{Definition of obesity}
\label{sub:definition_of_obesity}
% TODO: add WHO website to the reference for the definition of obesity (or find proper articles)

Obesity is defined as an abnormal or excessive fat accumulation that may impair health of that individual \citep{Garrow1988}.

One common and widely used approach to categorise individuals is by measuring the \gls{bmi} of the individual.
\gls{bmi} is a measurement based on the weight-to-height ratio of an individual that is used by clinicians and epidemiologists to classify adults into underweight, normal weight, overweight or obese categories.
The unit of \gls{bmi} is defined as the individual's weight in kilograms per square of the height in meters (kg/m$^2$).

\gls{who} (2014) have used \gls{bmi} measurements to define overweight and obesity in adults as an individual with \gls{bmi} $\geq$ 25 kg/m$^2$ and $\geq$ 30 kg/m$^2$, respectively.
For children under the age of 5, overweight and obesity were defined as weight-for-height greater than 2 standard deviation and 3 standard deviation above \gls{who} Child Growth Standards median, respectively.
For children aged between 5 to 19 years, overweight and obesity were defined as BMI-for-age greater than 1 standard deviation and 2 standard deviation above \gls{who} Growth Reference Standards median, respectively.

Although the use of \gls{bmi} as an accurate measure of body fat composition and/or representation of the individual's metabolic status remains controversial, many studies have used \gls{bmi} due to the ease of data collection compared to other measurements, such as [example] (citation needed).
Consequently, most of the publicly available clinical data have height and weight data from which one can calculate the patients' \gls{bmi}, and therefore able to collect the patients' obesity status more conveniently than, for example, [example].

possible papers:\citep{Gelber2008, Lee2008, Yusuf2005}

\subsection{Prevalence of obesity}
\label{sub:prevalence_of_obesity}

From the latest global status report on noncommunicable diseases by \citet{WHO2014}:
\begin{quote}
	\textit{
		In 2014, 39\% of adults aged 18 years and older (38\% of men and 40\% of women) were overweight.
		The worldwide prevalence of obesity nearly doubled between 1980 and 2014.
		In 2014, 11\% of men and 15\% of women worldwide were obese.
		Thus, more than half a billion adults worldwide are classed as obese.
	}
\end{quote}

\noindent
This equated to approximately 1 in every 14 people in the world classed as obese in 2014, and therefore, 1 in 14 people around the world were also at severe health risks associated with obesity.
It was also noted in the report that women were more likely to be obese than men, and that the prevalence of overweight and obesity increased as the income level of the countries increased \citep{WHO2014}.

In addition to this, the worldwide prevalence of childhood overweight and obesity has been increasing steadily since 2000 \citep{WHO2014}.
Furthermore, the prevalence of overweight in children under 5 years was estimated to reach from 6.3\% in 2013 to 11\% by 2025 if the current trend continued \citep{WHO2014}.
The current obesity status of the adult population is bad already, but the fact that more of the younger population who are overweight and obese poses a much greater risk for the health of the population (discussed in \cref{sub:impact_of_obesity}).

Until these terrible trends stop, and even after the trends stop, the world will need a plan to live along with this ongoing global epidemic of obesity and its associated diseases.

\subsection{Physiological mechanism of obesity}
\label{sub:physiological_mechanism_of_obesity}

Obesity is caused by continuous intake of food, whether they are controlled or not, without expending all of the energy gained.
The key to the problem is the control of food intake and energy expenditure, and how they are regulated in your body.

\subsubsection{Role of hypothalamus and arcuate nucleus in food intake control}
\label{ssub:role_of_hypothalamus_and_arcuate_nucleus_in_food_intake_control}

The central role of hypothalamus in the context of obesity is the control of satiety, and therefore food intake.
The hypothalamus receives both nueronal and hormonal inputs from the body and propagates the signal to the downstream neurons that ultimately affect the food intake of the individual \citep{Bell2005, Spiegelman2001}.

Within the hypothalamus, there is a group of neurons located within the arcuate nucleus that is essential for satiety signal transmission.
There are two types of neurons in the arcuate nucleus: orexogenic and anorexogenic neurons.
Orexogenic neurons are responsible for the promotion of food intake and reducing energy expenditure, whereas the anorexogenic neurons have the opposite effect \citep{Barsh2002}.

The orexogenic neurons produce \gls{npy} and \gls{agrp} which induces orexogenic signal to the downstream neurons; anorexogenic neurons produce \gls{pomc} and \gls{cart} which induces anorexogenic signal to the downstream neurons \citep{Barsh2002, Spiegelman2001}.
When stimulated by, for example, endocrine signals, these two different types of neurons act on the downstream neurons to promote or reduce food intake and energy expenditure.

\subsubsection{Leptin-melanocortin pathway}
\label{ssub:leptin_melanocortin_pathway}

Leptin-melanocortin pathway is the pathway in which the level of satiety is controlled and regulated.
In this pathway, both the orexogenic and anorexogenic neurons have crucial roles in controlling the satiety of the individual, and therefore their eating behaviour.

When a satiety-controlling hormone, such as leptin, reaches the arcuate nucleus region of the hypothalamus, it can stimulate or inhibit orexogenic, anorexogenic, or both orexogenic and anorexogenic neurons.
Depending on which neurons are stimulated or inhibited, the hormone will ultimately alter the satiety level of the individual.

Firstly, when an orexogenic \gls{npy}/\gls{agrp} neurons are stimulated, both \gls{npy} and \gls{agrp} are produced in the cell.
Both \gls{npy} and \gls{agrp} are neuropeptides that sends a signal to increase food intake and decrease energy expenditure.
\gls{npy} was first discovered by \citet{Tatemoto1982}, and later associated with altered eating behaviour by \citet{Clark1984}.
Agouti protein is a protein normally expressed in the skin where it affects the pigmentation, and was first observed over 100 years ago by \citet{Castle1910} in mice that had yellow, or agouti-coloured coat.

It was later found the ubiquitous expression of the protein was caused obesity (citation)


The \gls{npy} and \gls{agrp} neuropeptides are recognised by the downstream neurons via the \gls{nyr} and \gls{mcr}.

\subsubsection{Role of hormones in leptin-melanocortin pathway}
\label{ssub:role_of_hormones_in_leptin_melanocortin_pathway}

There are many hormones in your body that regulates various physiological aspects of your body.
Out of the many, four hormones have an important role in regulating the satiety of an individual: ghrelin, leptin, insulin, and \gls{pyy}.

Ghrelin is a hormone produced mainly in the stomach that has a role in meal initiation and food intake \citep{Kohno2003}.

Leptin, on the otherhand, is a hormone that has an opposite effect as ghrelin, where the hormone signals satiety and reduces food intake.

Insulin, perhaps the most well-known hormone for its role in glucose homeostasis, also has an important role in satiety and food intake regulation.

The last hormone is the \gls{pyy}, a peptide hormone released by the distal gastrointestinal tract that produces an anorexogenic signal.
\citep{Batterham2002}


\subsection{Factors that affect the prevalence of obesity}
\label{sub:factors_that_affect_the_prevalence_of_obesity}

As one can suspect, there are many factors involved with the likelihood of an individual to become overweight and obese.

\subsubsection{Individual-level (micro-level) factors}
\label{ssub:Individual-level_(micro-level)_factors}

There are many environmental factors associated with the global rise in obesity (discussed in \cref{ssub:Population-level_(macro-level)_factors}: \nameref{ssub:Population-level_(macro-level)_factors}).
Even though these environmental factors are probably the most significant contributor of the current global obesity epidemic, there are some individual-level, genetic factors that assist in the overall increase in the obese population.



There are many genetic evidence of obesity being caused by individual-level factors that makes some individuals more predisposed (right word??) to obesity than others (citation).


LEP, LEPR, POMC, CART, MC4R, etc and its associated genetic/biological mechanisms\\

\citep{Montague1997}
\citep{Clement1998}
\citep{Jackson1997}
\citep{Krude1998}
\citep{Farooqi2003}
\citep{Kublaoui2006}
\citep{Dubern2001}
\citep{Challis2002}

FTO and other important GWAS studies \\

\citep{Frayling2007}
\citep{Dina2007}
\citep{Scuteri2007}
\citep{Gerken2007}

Talk about all the mutations and contribution of these genes to obesity, and then link it back to ``thrifty'' gene
\citep{Neel1962}

\textit{Talk about individual-level factors that affect obesity here (need citations on individual factors of obesity; e.g. genetic variation).}

\subsubsection{Population-level (macro-level) factors}
\label{ssub:Population-level_(macro-level)_factors}

Even though some individual-level factors are involved with obesity, it is clear that these individual-level factors cannot be accounted for the global epidemic of obesity that the world currently observes.
For this reason, obesity is thought to be caused mainly by the environmental factors that are present in the population.\\

\noindent
The first key factor that contirbutes to the global rise in obese population is the availability and abundance of certain food in the country.
With the help from government trade policies, food and other goods have become easier to trade between the country, helping the economic growth of many countries around the world \citep{Kearney2010}.
This led to greater abundance of food in general, greater number of choices of food and shifted the overall nutritional status of the country \citep{Malik2013}.
This shift in the food choice and availability (and the economic benefit associated with the trades) had a positive impact on the country, but the growth in the food market also resulted in a different, much larger problem -- obesity.

As an example, in the \gls{usa}, the cost of corn and soy is low due to the fact that they are the raw ingredients for most processed food and beverages, such as high fructose corn syrups used to sweeten many soft drinks in the \gls{usa} \citep{Malik2013}.
Moreover, corn and soy are also the main food for the livestock, which results in lower prices for meat \citep{Malik2013}.
On the othere hand, the prices of fruits and vegetables remain expensive due to the lack of support by the government to lower the cost associated with the production \citep{Malik2013}.

As a result, more people are inclined to consume food that are cheap, have little nutritional value and high in energy, than the food that are expensive, nutritional and healthy alternatives.
Thus, the population is more likely to become obese by making worst food choice.\\

\noindent
The second factor to consider is the behavioural changes led by the urbanisation of the population.
Urbanisation is defined as the gradual increase in the proportion of people living in urban areas.
As more people take up urban lifestyle, they are exposed to the new environment where there is a better range of food selection, as well as technological and mechanical advancement in transport and other daily chores that may not be available in rural areas where these people come from.

Although there are many advantages involved with urbanisation, such as access to developed health care systems, education and advanced technologies, the lifestyle change also impose negative health behaviour that ultimately lead to positive energy balance and therefore obesity \citep{Malik2013}.

Lack of physical activity is one of the obvious components that contribute to obesity, which is also linked with the urbanisation of the population.
Currently, the recommended amount of physical activity to keep an individual fit and healthy is $\geq$ 150 min of moderate-intensity aerobic physical activity per week, or $\geq$ 30 min of physical activity on most days of the week \citep{Pate1995, WHO2010}.
However, in many urban cities, this recommended level of physical exercise is difficult to achieve and maintain, due to the shift towards the sedentary behaviours that are encouraged by the developed transport system, automated household chores, and indoor entertainment \citep{Malik2013}.

Urbanisation by itself cannot be accounted for the lack of exercise in the population, but it is true that urbanisation makes the bar higher for many people to take up the recommended level of fitness and maintain it. \\

\noindent
The third factor is the income and socioeconomic status of the population.
As the average income increases, more people are able to buy food and are likely to take up the sedentary lifestyle.
Clearly, these behavioural changes will increase the likelihood of becoming obese.

From this, one would expect the high-income group to have the largest impact from obesity, but on the contrary to this expectation, the biggest impact from obesity is seen in the low- and middle-income groups of the population \citep{Malik2013}.
As mentioned briefly earlier, access to the developed health care system will allow weight maintenance of the population, but this is likely to be of most benefit for the high-income group in the population.
The reason for this is, quite simply, the cost associated with the visit to the health care system.
and therefore high-income group are more likely to make a visit than the low or middle-income group of the population \citep{Malik2013}.

So, for the low- and middle-income groups, the rise in average income in these groups allow the people to have access to food and activities that promotes sedentary bahaviours, but the limited access to the health care system, education and recreational activities that promote physical activites ultimately leads to positive energy balance and obesity \citep{Malik2013}.
In contrast, the high-income group of the population have better access to the facilities that promote weight maintenance, thus have less of an impact from obesity compared with the low- and middle-income groups of the population.\\

\noindent
The last factor, which ties in strongly with the first key factor, is the choice of diet made by the population.
The amount of low-quality food have increased significantly in many rapidly growing countries, as these high-energy, low-nutritional food are cheaper and more profitible than the natural produce that have higher nutritional value \citep{Kearney2010}.
As a result, more people tend to purchase these low-quality food more often, and consume more energy than they are required.

However, many people eat responsibly and avoid these low-quality diet and prevent overconsumption of such food, but the overconsumption of these food is not a big problem as such, rather the overall quality of the food \citep{Malik2013}.
These food are usually high in refined sugar and unhealthy fat, highly processed, contain very little nutritional value and are low in fibre.
As a result, consumption of these food increases the overall energy intake and achieve no nutritional benefits from the food.

Whether it is by choice or not, consumption of these low-quality diet leads to obesity and obesity-related diseases. \\

% All of these population-level, factors together with the individual-level factors, contribute to the obesity status of the population.


\subsection{Impact of obesity}
\label{sub:impact_of_obesity}

As mentioned at the beginning of the current section, obesity has a devastating effect on the health of the population in a global scale.


\citep{Malik2013, Franks2010}




\subsection{Preventive measures of obesity}
\label{sub:preventive_measures_of_obesity}




\citep{Malik2013}




\section{Cancer}
\label{sec:cancer}

% What is cancer.

\subsection{Hallmarks of Cancer}
\label{subsec:cancerhallmarks}

% TODO: edit the starting sentence so it's not a copy from the beginning
% It is now commonly understood that cancers are caused by dysregulation of various mechanisms or pathways that allow tumour cells to prolilferate, survive and migrate.
% Many of these dysregulated pathways frequently occur in cancer cells, and these characteristic traits, commonly known as the ``Hallmarks of Cancer'', were first described by \citet{Hanahan2000}.

% These hallmarks were initially presented in 2000 to provide a common framework for scientists to better understand cancer biology, and also to direct the focus of future research.
% After a decade of advances in cancer research, the ``Hallmarks of Cancer'' were revisited by Hanahan and Weinberg in 2011, and four more hallmarks were added on top of the six presented in the original paper in 2000 \citep{Hanahan2011}.

% Together, these ``Hallmarks of Cancer'' (fig 1) have provided a foundation of the general biological knowledge of cancer and its characteristics that allowed the researchers to better understand the underlying cancer biology.

% Although much reseach has been conducted on the ``Hallmarks of Cancer'', the molecular processes underlying these hallmarks have still not been fully explained \citep{Hanahan2011}.
% This is due to the fact that cancer cells develop progressively under selective pressures, and are also affected by many environmental factors \citep{Gatza2011}, which makes it difficult to identify causal factors of cancer development.
% Furthermore, these external factors affect how the cancer cells respond to the environment, and may also cause cancer progression and/or drug resistance.

\subsection{Tumour heterogeneity}
\label{sub:tumour_heterogeneity}




\section{Obesity and cancers}
\label{sec:obesity_and_cancers}

\subsection{Cancer risks associated with obesity}
\label{sub:cancer_risks_associated_with_obesity}

\citep{Calle2003}

\subsection{Mechanism of cancer progression in obese patient}
\label{sub:mechanism_of_cancer_progression_in_obese_patient}



% Why is it important in obesity context.

% The exact mechanism in which obesity accelerates cancer progression is unknown, but obesity is thought to disrupt the levels of various hormones (such as insulin, leptin, adeponectin, and other adipokines) that eventually activate tumorigenic pathways (reference the overview on obesity and tumour resistance).

\section{Genetic signatures}
\label{sec:genetic_signatures}

\subsection{Microarray technology}
\label{subsec:microarray_technology}

\subsection{Next generation sequencing technology}
\label{sub:next_generation sequencing_technology}



% Due to advances in sequencing technology over the last few years, many researchers have started to focus on bioinformatics based analyses of cancer cells to better understand the biology of cancer.
% Advantages of this strategy are that very specific gene expression and/or mutational signatures for certain cancer subtypes can be identified from the analyses, which the researchers and clinicians can study in much greater depth and details.

% TODO: have a leading sentence that introduces the gene signatures

\section{Obesity associated genetic signatures}
\label{sec:obesity_associated_genetic_signatures}

\subsection{\citet{Creighton2012} study}
\label{sub:creighton_study}



\subsection{\citet{Fuentes-Mattei2014} study}
\label{sub:fuentes_mattei_study}




\section{Pathway associated genetic signatures}
\label{sec:pathway_associated_genetic_signatures}

% Talk about \citet{Gatza2010a}

\subsection{\citep{Gatza2011} study}
\label{sub:gatza_study}





\section{Aim of the project}
\label{sec:aim}

This research aims to determine whether gene expression signatures exist  that are specific to obesity across multiple cancer types, and to investigate whether there are any common pathways being dysregulated in cancers based on these genetic signatures.
Better understanding of the pathways being dysregulated in cancer cells in obese patients may lead to improved clinical decisions, and contribute towards personalised treatment in the future.

