\chapter{Results from \citet{Creighton2012} metagene (draft)}

\section{Validation of previously identified obesity-associated genetic signatures}

(draft)\\
To validate the ability for the obesity-associated biomarkers to categorise the obese samples from non-obese samples, singular value decomposition with the previously identified 799 gene probes was used on the raw data.
The metagene produced from this was used to check whether the obesity-associated biomarkers were able to identify the difference in obesity groups within the samples.
(insert first figure - gene expression sorted by metagene)
The resulting heatmap shown in figure~(fig:crobsheat) shows how the overall gene expression of each sample is represented well by the metagene created from the 799 gene probes identified by \citet{Creighton2012}.

To assess whether the metagene was actually correlating with the sample BMI values and statuses, a scatter plot and a box plot were plotted against the metagene, respectively.
These plots proved that the obesity-associated signatures identified by \citet{Creighton2012} were able to classify the samples based on their BMI status (specifically the obese samples from the non-obese samples), confirming the results fromthe study.

\section{Reproducing the obesity-associated genetic signatures}

(draft - are we able to find the same/similar genes as creighton, using the same data?)\\
To see whether we were able to re-identify the same, or similar, gene probes as \citet{Creighton2012}, differential gene expression analyses were carried out between the obese and non-obese samples, with different statistical conditions.

\section{Application of the obesity-associated genetic signatures to other cancer types}

(draft - found nothing)

\section{\citet{Fuentes-Mattei2014} gene signatures on \citet{Creighton2012} data and vice versa}

(draft)

\section{\citet{Fuentes-Mattei2014} gene signatures on TCGA data and vice versa}

(draft)

