%%%%%%%%%%%%%%%%%%%%%%%%%%%%%%%%%%%%%%%%%%%%%%%%%%%%%%%%%%%%
% Abstract for my thesis
%%%%%%%%%%%%%%%%%%%%%%%%%%%%%%%%%%%%%%%%%%%%%%%%%%%%%%%%%%%%
\vspace*{\fill}

\section*{\centering Abstract}
\addcontentsline{toc}{section}{Abstract}

It is now commonly understood that cancers are caused by dysregulation of various mechanisms or pathways that allow tumour cells to prolilferate, survive and migrate.
There have been many reports on the association between obesity and worsened cancer prognosis, suggesting that there may be distinct changes that underlie tumorigenesis and/or cancer progression that are specific to patients with higher \gls{bmi}.

In brief, this research aims to determine whether gene expression signatures exist that are specific to obesity across multiple cancer types, and to investigate whether there are any common pathways being dysregulated in cancers based on these genetic signatures.
Better understanding of the pathways being dysregulated in cancer cells in obese patients may lead to improved clinical decisions, and could thus contribute towards personalised treatment in the future.

\vfill
