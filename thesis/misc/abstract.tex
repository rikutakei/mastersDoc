%%%%%%%%%%%%%%%%%%%%%%%%%%%%%%%%%%%%%%%%%%%%%%%%%%%%%%%%%%%%
% Abstract for my thesis
%%%%%%%%%%%%%%%%%%%%%%%%%%%%%%%%%%%%%%%%%%%%%%%%%%%%%%%%%%%%
\vspace*{\fill}

\section*{\centering Abstract}
\addcontentsline{toc}{section}{Abstract}

Obesity has been a major global problem for more than a decade, associated with many noncommunicable diseases such as  cancer.
The number of obese people, both adults and children, has risen in every country of the world and the trend will likely to continue.
Cancers are caused by dysregulation of various molecular pathways that allow tumour cells to proliferate, survive and migrate.
One of the difficulties associated with the treatment of cancers is the identification of the underlying biological pathways that drive tumorigenesis.
This research aims to determine whether gene expression signatures exist  that are specific to obesity across multiple cancer types, and to investigate whether there are any common pathways being dysregulated in cancers based on these genetic signatures.
In this work no genetic signatures or differentially expressed genes were found between obese and non-obese patients that were common across multiple cancer types.
However, the Akt, \gls{egfr}, \gls{tgfb} and Src pathways may have a role in promoting the tumour progression in patients that are obese.
It is likely that there is some complex mechanism underlying the relationship between obesity and cancer.
A better understanding of the pathways being dysregulated in cancer cells in obese patients may lead to improved clinical decisions, and contribute towards personalised treatment in the future.

\vfill
\vfill
