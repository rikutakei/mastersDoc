\chapter{Methods (draft)}
\label{ch:methods}

\section{\gls{bmi}}
\label{sec:bmi}

\subsection{\gls{bmi} Calculation}
\label{subsec:bmicalc}

\gls{bmi} was calculated from the height and weight data of the samples using the following equation:

\begin{equation}
	\label{eq:bmicalc}
	\gls{bmi} = \frac{Weight (kg)}{Height^2(m^2)}\\
\end{equation}

\subsection{\gls{bmi} Classification}
\label{subsec:bmiclassification}

Samples were classified based on the \gls{who} definition, as shown in Table~\cref{tab:whobmiclass}.
\begin{table}[hb]
	\caption{\gls{who} defined \gls{bmi} classification}
	\label{tab:whobmiclass}
	\begin{center}
		\begin{tabular}{lc}
			\bfseries {Classification} & \bfseries {\gls{bmi} Value}\\
			\hline
			\rule{0pt}{2.25ex}Underweight & \textless{} 20.0\\
			Normal weight/lean & 20.0$\sim$24.9\\
			Overweight & 25.0$\sim$29.9\\
			Obese & \textgreater{} 30.0\\
		\end{tabular}
	\end{center}
\end{table}

\section{Publicly available cancer data}
\label{sec:data}

Data related to multiple different cancer types were downloaded from free and publicly available sources such as \gls{tcga} and the \gls{icgc}.

\subsection{\gls{tcga}}
\label{subsec:tcga}

The clinical data for all available cancer types (33 types in total) were downloaded from \gls{tcga} database and were checked for the height and weight data for each sample, for each cancer type.
Any cancer type with no weight and/or height data of the samples were excluded from the project, as no \gls{bmi} information can be obtained without these data.

This gave a total of eight cancer types: \gls{blca}, \gls{cesc}, \gls{coad}, \gls{kirp}, \gls{lihc}, \gls{read}, \gls{skcm}, \gls{ucec}.
All eight cancer types were used for further analyses.

\subsection{\gls{icgc}}
\label{subsec:icgc}

\gls{rnaseq} data for the eight cancer types with sample \gls{bmi} information available were downloaded from the \gls{icgc} database.
The \gls{rnaseq} data for each of the eight cancer types were imported into \gls{r}.
%TODO: add in how I processed the ICGC data format into genes by samples.

%TODO: change the title name?
\section{\gls{icgc} sample exclusion criteria}
\label{sec:sampleexclusion}

Samples from the downloaded data were removed using the following criteria.

\subsection{Sample BMI availability}
\label{subsec:samplebmiavailability}

Some of the samples did not have either height or weight data, or both.
These samples were removed from the analyses.

\subsection{Sample ID conversion and matching}
\label{subsec:sampleidconversion}

%TODO: check the actual ID stripping process
%TODO: check the WGCNA function and check if it uses median/mean/whatever for choosing the sample

\gls{icgc} sample ID in the \gls{rnaseq} data were stripped so that the ID from the \gls{rnaseq} data matched with the \gls{tcga} ID in the clinical data.
After stripping the ID, samples were checked for any overlaps.
Where there was an overlap in the sample ID in the \gls{rnaseq} data, a single sample was chosen to represent that particular ID by using the \texttt{collapseRows} function in the \textit{WGCNA} package in \gls{r}.

In each of the eight cancer types, the \gls{tcga} sample ID in the clinical data were cross-checked with the \gls{tcga} sample ID in the \gls{rnaseq} data, and vice versa.
Any sample that did not have either clinical or \gls{rnaseq} data were removed, so both clinical and \gls{rnaseq} data were available for all of the samples that were included in the analyses.
See \cref{tab:samplesize} for a summary of the number of samples included in the analyses for each cancer type.

\begin{table}[h]
	\caption{Summary of the total number of samples included in the analyses for each cancer type}
	\label{tab:samplesize}
	\begin{center}
		\begin{tabular}{cc}
			\textbf{Cancer Type}   & \textbf{Total} \\
			\hline
			\rule{0pt}{2.25ex}BLCA & 261   \\
			CESC                   & 224   \\
			COAD                   & 226   \\
			KIRP                   & 124   \\
			LIHC                   & 264   \\
			READ                   & 73    \\
			SKCM                   & 218   \\
			UCEC                   & 482   \\
		\end{tabular}
	\end{center}
\end{table}

\section{Data processing -- Creighton data}
\label{sec:dataprocessingcr}

\subsection{Raw data from the \citet{Creighton2012} study}
\label{subsec:rawdatacr}

The raw microarray gene expression data files of all *** samples from the \citet{Creighton2012} study were downloaded from the \gls{geo} database (\gls{geo} accession ID: GSE24185).
Clinical data of the samples were obtained from the supplementary table 1 from their paper.
Raw data were imported into \gls{r}/RStudio for subsequent analyses.


%TODO: add data/sample composition -- ethinicity, grade, etc -- maybe in table form

\subsection{Normalisation of raw Creighton data}
\label{subsec:normcrdata}

The raw microarray data for all 103 samples from the study were normalised using \gls{rma} method from the \textit{affy} package in \gls{r}.
The data was then standardised so that each of the genes had a mean ($\mu$) of 0 and a standard deviation ($\sigma$) of 1.
This ensured that all of the genes were on the same scale as one another, allowing for comparison.

\subsection{Obesity associated gene probes from the \citet{Creighton2012} study}
\label{subsec:crobsgene}

799 obesity associated gene probes identified by \citet{Creighton2012} were obtained from the supplementary data file 1 from their paper.
The list of gene probes were imported into \gls{r}.
The gene probe IDs were converted into gene symbols using the \textit{hgu133a.db} package in \gls{r}.
Any duplicates and any gene symbols that were not present in both the Creighton data and the \gls{icgc} data were removed, which left *** unique obesity associated genes.

\section{Validation of results -- \citet{Creighton2012}}
\label{sec:valresultscr}

The results presented by \citet{Creighton2012} were validated using the same raw data they have used in their study.
See  \cref{sec:dataprocessingcr} for details on how the raw data was accessed and processed.

\subsection{Metagene and sample gene expression pattern}
\label{subsec:crmetage}

The gene probe IDs were converted into gene symbols using the \textit{hgu133a.db} package in \gls{r}.
Where there was a duplicated gene in the data, appropriate gene data was chosen to represent that particular gene by using the \texttt{collapseRows} function in the \textit{WGCNA} package in \gls{r}.
All of the *** obesity associated genes from  \cref{subsec:crobsgene} were selected from the data.

\gls{svd} was applied to this data to obtain the decomposed matrices of the data.
%TODO: how do I explain how I got the \gls{metagene} scores?
The \gls{metagene} scores of the samples were obtained by taking the first eigenvector of the (orthogonal matrix...?).
The \gls{metagene} scores were ranked based on the number of samples in the data (103 samples in Creighton data), and the values of the scores were between 0 and 1.

These \gls{metagene} scores were plotted together with the gene expression of the obesity associated gene probes for each sample on a heatmap.
The samples in the heatmap were ordered based on their ranked \gls{metagene} scores.
To have a better colour scales in the heatmap, any expression values that were greater than 3 or less than -3 were assigned the value of 3 and -3, respectively.

\subsection{Metagene and sample \gls{bmi}}
\label{subsec:crmetabmi}

The same \gls{metagene} scores as in the previous section (\cref{subsec:crmetage}) were used to compare the \gls{metagene} scores with the samples' \gls{bmi} statuses and values.
Boxplot was plotted to show the association of the \gls{metagene} scores with the sample \gls{bmi} status, and scatter plot was plotted to show the association with the sample \gls{bmi} value.

%TODO: check the formal definition of these test/values
For the boxplot, the p-value was obtained from the ANOVA test.
For the scatter plot, the regression line was plotted and the adjusted $R^2$ value was calculated for the regression line.


\section{Creighton metagene in \gls{icgc} cancer data}
\label{sec:crmetaicgc}

\subsection{\gls{icgc} data processing}
\label{subsec:datprocicgc}

The obesity associated genes from \cref{subsec:crobsgene} were selected in each of the eight cancer types.
All cancer data had 1 added to the count data, so no infinite values were introduced when the data was logged.
All cancer data were logged to the base 10.

All of the data were then standardised so that each of the genes had a mean ($\mu$) of 0 and a standard deviation ($\sigma$) of 1.
Any ``NaN'' values in the data introduced from this were assigned a value of 0.

\subsection{Transformation matrix}
\label{subsec:transmatcr}

Transformation matrix was formed from the decomposed matrices that resulted from the application of \gls{svd} to the Creighton data (see \cref{subsec:crmetage}).
%TODO: add the derivation of transformation matrix??

\subsection{Transformed \gls{icgc} data and metagene}
\label{subsec:icgctranscr}

Each of the eight cancer data from \cref{subsec:datprocicgc} were transformed using the transformation matrix from \cref{subsec:transmatcr}.
Metagene scores of the samples in each cancer types were obtained by taking the first column of the cancer matrix that resulted from the transformation.
The \gls{metagene} scores were ranked based on the number of samples in the data, and the scores ranged from 0 to 1.

\subsection{Creighton metagene and sample gene expression in \gls{icgc} data}
\label{subsec:crmetaicgcge}

\Gls{metagene} scores from \cref{subsec:icgctranscr} were plotted together with the gene expression of the obesity associated gene probes for each sample on a heatmap for each cancer type.
The data from \cref{subsec:datprocicgc} were used for the heatmap.
The samples in the heatmap were ordered based on their ranked \gls{metagene} scores.
To have a better colour scales in the heatmap, any expression values that were greater than 3 or less than -3 were assigned the value of 3 and -3, respectively.

\subsection{Creighton metagene and sample \gls{bmi} in \gls{icgc} data}
\label{subsec:crmetaicgcge}

The same \gls{metagene} scores as in the previous section were used to compare the \gls{metagene} scores with the samples' \gls{bmi} statuses and values for each cancer type.
Boxplot was plotted to show the association of the \gls{metagene} scores with the sample \gls{bmi} status, and scatter plot was plotted to show the association with the sample \gls{bmi} value.

%TODO: check the formal definition of these test/values
For the boxplot, the p-value was obtained from the ANOVA test.
For the scatter plot, the regression line was plotted and the adjusted $R^2$ value was calculated for the regression line.


%TODO: Not sure if I should cut this section to "see previous section for details"
\section{Data processing -- Fuentes-Mattei data}
\label{sec:dataprocessingfm}

\subsection{Raw data from the \citet{Fuentes-Mattei2014} study}
\label{subsec:rawdatafm}

The raw microarray gene expression data files of all *** samples from the  \citet{Fuentes-Mattei2014} study were downloaded from the \gls{geo} database (include website name/GEO ID here).
Clinical data for the samples in this study was not available.
Raw data were imported into \gls{r}/RStudio for subsequent analyses.

%TODO: add data/sample composition -- ethinicity, grade, etc -- maybe in table form

\subsection{Normalisation of raw Fuentes-Mattei data}
\label{subsec:normfmdata}

The raw microarray data for all *** samples from the study were normalised using \gls{rma} method from the \textit{affy} package in \gls{r}.
The data was then standardised so that each of the genes had a mean ($\mu$) of 0 and a standard deviation ($\sigma$) of 1.

\subsection{Obesity associated gene probes from the \citet{Fuentes-Mattei2014} study}
\label{subsec:fmobsgene}

130 obesity associated gene probes identified by \citet{Fuentes-Mattei2014} were downloaded from (supplementary data??).
The list of gene probes were imported into \gls{r}.
Any gene symbols that were not present in the Fuentes-Mattei data, the Creighton data and the \gls{icgc} data were removed, which left *** unique obesity associated genes.

\section{Validation of results -- \citet{Fuentes-Mattei2014}}
\label{sec:valresultsfm}

The results presented by \citet{Fuentes-Mattei2014} were validated using the Creighton data, as the clinical data of the samples were not available.
See \cref{subsec:rawdatacr,subsec:normcrdata} for details on how the data was processed.

\subsection{Fuentes-Mattei transformation matrix}
\label{subsec:transmatfm}

\gls{svd} was applied to Fuentes-Mattei data to obtain the decomposed matrices of the data, and this was used to make the transformation matrix.
%TODO: add the derivation of transformation matrix??

\subsection{Fuentes-Mattei metagene and gene expression pattern in Creighton data}
\label{subsec:fmmetage}

The gene probe IDs in Creighton data were converted into gene symbols using the \textit{hgu133a.db} package in \gls{r}.
Where there was a duplicated gene, appropriate gene data was chosen to represent that particular gene by using the \texttt{collapseRows} function in the \textit{WGCNA} package in \gls{r}.
All of the *** obesity associated genes from  \cref{subsec:fmobsgene} were selected from the data.

%TODO: how do I explain how I got the \gls{metagene} scores?
The \gls{metagene} scores of the samples were obtained by taking the first eigenvector of the transformed Creighton data.
The \gls{metagene} scores were ranked based on the number of samples in the data (*** samples in Creighton data), and the scores ranged from 0 to 1.

These \gls{metagene} scores were plotted together with the gene expression of the obesity associated gene probes for each sample on a heatmap.
The samples in the heatmap were ordered based on their ranked \gls{metagene} scores.
To have a better colour scales in the heatmap, any expression values that were greater than 3 or less than -3 were assigned the value of 3 and -3, respectively.

\subsection{Metagene and sample \gls{bmi}}
\label{subsec:fmmetabmi}

The same \gls{metagene} scores from the previous section were used to compare the \gls{metagene} scores with the samples' \gls{bmi} statuses and values.
Box and scatter plots, and any statistical values associated with these plots were calculated as in \cref{subsec:crmetaicgcge}.


\section{Fuentes-Mattei metagene in \gls{icgc} cancer data}
\label{sec:fmmetaicgc}

\subsection{\gls{icgc} data processing}
\label{subsec:datprocicgcfm}

All of the \gls{icgc} data were processed in a similar fashion as in \cref{subsec:datprocicgc}, but the obesity associated genes from \cref{subsec:fmobsgene} were used instead.

\subsection{Transformation matrix}
\label{subsec:icgctransmatfm}

The same transformation matrix from \cref{subsec:transmatfm} was used.

\subsection{Transformed \gls{icgc} data and metagene}
\label{subsec:icgctransfm}

Each of the eight cancer data from \cref{subsec:datprocicgcfm} were transformed using the transformation matrix from \cref{subsec:transmatfm}.
Metagene scores were ranked as in \cref{subsec:icgctranscr}.

\subsection{Fuentes-Mattei metagene and sample gene expression in \gls{icgc} data}
\label{subsec:fmmetaicgcge}

\Gls{metagene} scores from \cref{subsec:icgctransfm} were plotted together with the gene expression of the obesity associated gene probes for each sample on a heatmap for each cancer type.
The data from \cref{subsec:datprocicgcfm} were used for the heatmap.
The samples in the heatmap were ordered based on their ranked \gls{metagene} scores.
To have a better colour scales in the heatmap, any expression values that were greater than 3 or less than -3 were assigned the value of 3 and -3, respectively.

\subsection{Creighton metagene and sample \gls{bmi} in \gls{icgc} data}
\label{subsec:fmmetaicgcge}

The same \gls{metagene} scores as in the previous section were used to compare the \gls{metagene} scores with the samples' \gls{bmi} statuses and values for each cancer type.
Box and scatter plots, and any statistical values associated with these plots were calculated as in \cref{subsec:crmetaicgcge}.

\section{Differential gene expression analysis}
\label{sec:deg}

\subsection{Data processing and subsetting}
\label{subsec:subdatadeg}

Raw data from \cref{subsec:rawdatacr,subsec:normcrdata} were used in the analysis.
Together with the raw data, few other data sets were created based on the raw data: residual data, Caucasian-only raw data, and Caucasian-only residual data.

Residual data was created from the raw data after it has been controlled for all clinical variables, except those related to \gls{bmi}.
These clinical variables included age, sex, ethinicity, tumour grade, \gls{pr}, \gls{er}, \gls{her2} statuses, and \gls{ln} status.

Caucasian-only data was created by taking only the Caucasian samples from the raw data, which reduced the sample size from 103 samples to 77 samples.
Caucasian-only residual data was created using the same approach as with the residual data: the data that remained after all the non-\gls{bmi} clinical variables were controlled in the Caucasian-only raw data.

\subsection{Group classification}
\label{subsec:crdegclass}

In each of the datasets from \cref{subsec:subdatadeg}, all of the samples were split into two groups: obese and non-obese groups.
These two groups were classified based on the sample \gls{bmi}, where the obese groups were defined as samples with \gls{bmi}  \textgreater{} 30, and non-obese group were defined as samples with \gls{bmi}  \textless{} 30 (see \cref{tab:whobmiclass}).

\subsection{Gene expression analysis}
\label{subsec:crdeg}

The gene probe IDs were converted into gene symbols using the \textit{hgu133a.db} package in \gls{r}.
Where there was a duplicated gene in the data, appropriate gene data was chosen to represent that particular gene by using the \texttt{collapseRows} function in the \textit{WGCNA} package in \gls{r}.






The difference in the gene expression between these two sample groups were analysed by fitting a linear model to the data, using the \textit{limma} package in \gls{r}.
The top 799 differentially expressed gene probes that had p-value \textless{} 0.01 were identified.





















\newpage



The process above was repeated on the same data, with the same group (obese and non-obese) after controlling for all other clinical variables that were not related to the sample \gls{bmi} (age, sex, ethinicity, tumour grade, \gls{ln} status, \gls{pr} status, \gls{er} status, \gls{her2} status).
the top 799 differentially expressed gene probes that had p-value \textless{} 0.01 between the two groups were identified.

To completely remove the effect of ethinicity on the analysis, the analysis was repeated using only the Caucasian samples.
This reduced the sample size from 103 samples to 77 samples.
The data was controlled for all other clinical variables except the variables that were related to \gls{bmi} (age, sex, tumour grade, \gls{ln} status, \gls{pr} status, \gls{er} status, and \gls{her2} status).
Again, the top 799 differentially expressed genes between the two groups with p-value \textless{} 0.01 were identified.

\subsection{Comparison of \citet{Creighton2012} gene probes and identified gene probes}
\label{subsec:cfgeneprobes}

The obesity associated gene probes identified by \citet{Creighton2012} were compared with each of the three sets of gene probes identified from the differential gene expression analysis (see \cref{sec:deg} \nameref{sec:deg}).
The presence of any of the \citet{Creighton2012} genes were checked for in each of the gene sets identified at a p-value \textless{} 0.05, as well as the top 799 gene probes at p-value \textless{} 0.01.

\subsection{\gls{svd}}
\label{subsec:svd}

(need svd derivation and why I'm using it)

\subsection{Correlation between obesity associated metagene and sample \gls{bmi}}
\label{subsec:metavsbmi}

In order to check whether the metagenes were actually specific to the \gls{bmi} status of the samples, the metagenes were plotted against both the \gls{bmi} statuses and the \gls{bmi} values of the samples.
The metagenes were ranked for each sample, and a box plot and a scatter plot were plotted for sample \gls{bmi} statuses and \gls{bmi} values, respectively.

\section{Applying metagene to other cancer types}
\label{sec:metagene}

(How you could transfer the metagene to other cancer data)

\subsection{Transformation matrix}
\label{subsec:transmat}

(why transformation matrix is used)
(refer back to eqiations in the  svd section)

\section{Pathway Enrichment Analysis}
\label{sec:pathenrich}







