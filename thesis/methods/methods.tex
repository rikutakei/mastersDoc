\chapter{Methods (draft)}
\label{ch:methods}

\section{\gls{bmi}}
\label{sec:bmi}

\subsection{\gls{bmi} Calculation}
\label{subsec:bmicalc}

\gls{bmi} was calculated from the height and weight data of the samples using the following equation:

\begin{equation}
	\label{eq:bmicalc}
	\gls{bmi} = \frac{Weight (kg)}{Height^2(m^2)}\\
\end{equation}

\subsection{\gls{bmi} Classification}
\label{subsec:bmiclassification}

Samples were classified based on the \gls{who} definition, as shown in \cref{tab:whobmiclass}.
\begin{table}[hb]
	\caption{\gls{who} defined \gls{bmi} classification}
	\label{tab:whobmiclass}
	\begin{center}
		\begin{tabular}{lc}
			\bfseries {Classification} & \bfseries {\gls{bmi} Value}\\
			\hline
			\rule{0pt}{2.25ex}Underweight & \textless{} 20.0\\
			Normal weight/lean & 20.0$\sim$24.9\\
			Overweight & 25.0$\sim$29.9\\
			Obese & \textgreater{} 30.0\\
		\end{tabular}
	\end{center}
\end{table}

\section{Publicly available cancer data}
\label{sec:data}

% TODO: add Cris Print's paper
The raw microarray data from \citet{Creighton2012}, \citet{Fuentes-Mattei2014} and \citet{Gatza2010a}  studies were downloaded from the \gls{geo} website.
\gls{rnaseq} and clinical data for multiple different cancer types were downloaded from free and publicly available sources such as \gls{icgc} and \gls{tcga}, respectively.

\subsection{Raw data from \citet{Creighton2012} study}
\label{subsec:crrawdat}

%TODO: add data/sample composition -- ethinicity, grade, etc -- maybe in table form

The raw microarray gene expression data files of all 103 samples from the \citet{Creighton2012} study were downloaded from the \gls{geo} database (\gls{geo} accession ID: GSE24185).
Clinical data of the samples were obtained from the supplementary table 1 from \citet{Creighton2012} paper.
799 obesity associated gene probes identified in the \citet{Creighton2012} study were obtained from the supplementary data file 1 from \citet{Creighton2012} paper.
The raw expression data, clinical data, and the list of gene probes were imported into \gls{r}/RStudio.

\subsection{Raw data from \citet{Fuentes-Mattei2014} study}
\label{subsec:fmrawdat}

%TODO: add data/sample composition -- ethinicity, grade, etc -- maybe in table form
%TODO: find out where  the genes were taken from

The raw microarray gene expression data files of all *** samples from the  \citet{Fuentes-Mattei2014} study were downloaded from the \gls{geo} database (include website name/GEO ID here).
Clinical data for the samples in this study was not available.
130 obesity associated gene probes identified by \citet{Fuentes-Mattei2014} were downloaded from (supplementary data??).
The raw expression data and the list of gene probes were imported into \gls{r}/RStudio.

\subsection{Raw data from Print et al study}
\label{subsec:printrawdat}

%TODO: add data/sample composition -- ethinicity, grade, etc -- maybe in table form
%TODO: find out where  the genes were taken from
%TODO: Change Fuentes-Mattei citation to Cris Print's citation

The raw microarray gene expression data files of all *** samples from the  \citet{Fuentes-Mattei2014} study were downloaded from the \gls{geo} database (include website name/GEO ID here).
Clinical data for the samples in this study was not available.
130 obesity associated gene probes identified by \citet{Fuentes-Mattei2014} were downloaded from (supplementary data??).
The raw expression data and the list of gene probes were imported into \gls{r}/RStudio.

\subsection{Raw data from \citet{Gatza2010a} study}
\label{sub:gatzarawdat}

%TODO: add data/sample composition -- ethinicity, grade, etc -- maybe in table form
%TODO: find out where  the genes were taken from
%TODO: Change Fuentes-Mattei citation to Cris Print's citation

The raw microarray gene expression data files used in  \citet{Gatza2010a} study were downloaded from the \gls{geo} database (include website name/GEO ID here).
Clinical data for the samples in this study was not available.
130 obesity associated gene probes identified by \citet{Gatza2010a} were downloaded from (supplementary data??).
The raw expression data and the list of gene probes were imported into \gls{r}/RStudio.

\subsection{\gls{tcga}}
\label{subsec:tcga}

The clinical data for all available cancer types (33 types in total) were downloaded from \gls{tcga} database and were checked for the height and weight data for each sample.
Any cancer type with no weight and/or height data of the samples were excluded from the project, as no \gls{bmi} information can be obtained without these data.

This gave a total of eight cancer types: \gls{blca}, \gls{cesc}, \gls{coad}, \gls{kirp}, \gls{lihc}, \gls{read}, \gls{skcm}, \gls{ucec}.
All eight cancer types were used for further analyses.

\subsection{\gls{icgc}}
\label{subsec:icgc}

\gls{rnaseq} data for the eight cancer types identified in \cref{subsec:tcga} were downloaded from the \gls{icgc} database.
The \gls{rnaseq} data for each of the eight cancer types were imported into \gls{r}/RStudio.





\section{Data processing}
\label{sec:datproc}

\subsection{Data normalisation}
\label{sub:data_normalisation}

\subsubsection{Microarray data}
\label{ssub:microarray_data}

The raw microarray data were normalised using either the \gls{rma} or \gls{mas} method from the \textit{affy} package in \gls{r}.
Either \gls{rma} or \gls{mas} normalisation was used to account for the inter-array differences that may have arisen during the data collection process.

\subsubsection{RNA-seq data}
\label{ssub:rna_seq_data}

The raw \gls{rnaseq} data for each cancer types downloaded from \gls{icgc} website had 1 added, and then logged to the power of 10.

\subsection{Data standardisation}
\label{sub:data_standardisation}

The data was standardised so that each gene in the data had a mean ($\mu$) of 0 and a standard deviation ($\sigma$) of 1.
This ensured that all of the genes were on the same scale as one another, allowing for comparison across all the genes.

\subsection{Residual data creation}
\label{sub:residual_data_creation}

To create the residual data for a given data set, a linear model was fitted to the data with all of the clinical variables that required to be controlled for.
The \textit{limma} package in \gls{r} was used.

\subsection{Gene probe ID conversion}
\label{sub:gene_probe_id_conversion}

Where appropriate, the gene probe IDs in the raw data were converted into their corresponding gene symbols using the \textit{hgu133a.db} package in \gls{r}.
Where there was an overlap in the gene symbols, a single data was chosen to represent that particular gene symbol by using the \texttt{collapseRows} function in the \textit{WGCNA} package in \gls{r}.
Likewise, the obesity associated gene probes were also converted into gene symbols where required.


\subsection{Batch correction}
\label{sub:batch_correction}

When there were more than one microarray data to be used as a single data set, these microarray data were normalised separately then combined together with the \texttt{ComBat} function in the \textit{sva} package in \gls{r}.
This corrected for the effect of the individual microarray data on the whole data when they were combined together.

\subsection{Sample \gls{bmi} calculation}
\label{sub:sample_bmi_calculation}

Where missing, the \gls{bmi} of the samples were calculated from the clinical data using \cref{eq:bmicalc}.

\subsection{\gls{icgc} data format conversion}
\label{sub:icgc_data_format_conversion}

The raw \gls{rnaseq} data for each cancer types were formatted using the \textit{dplyr} package in \gls{r}, so that the rows corresponded to the genes and the columns to the samples.

\subsection{Sample ID conversion}
\label{sub:sample_id_conversion}

\gls{icgc} sample ID in the \gls{rnaseq} data were stripped so that the sample ID matched with the \gls{tcga} ID in the clinical data.
After the IDs were stripped, samples were checked for any duplicates.
Where there was a duplicate in the sample ID in the \gls{rnaseq} data, a single sample was chosen to represent that particular ID by using the \texttt{collapseRows} function in the \textit{WGCNA} package in \gls{r}.

\subsection{\gls{icgc} sample exclusion}
\label{sub:icgc_sample_exclusion}

Some of the samples did not have either height or weight data, or both.
These samples were removed from the analyses.
In each of the eight cancer types, the \gls{tcga} sample ID in the clinical data were cross-checked with the sample ID in the \gls{rnaseq} data, and vice versa.
Any sample that did not have either clinical or \gls{rnaseq} data were removed.
This ensured that both clinical and \gls{rnaseq} data were available for all of the samples that were included in the analyses.
See \cref{tab:samplesize} for a summary of the number of samples included in the analyses for each cancer type.

%TODO: May need more details on the sample (bmi status, etc.)
\begin{table}[h]
	\caption{Summary of the total number of samples included in the analyses for each cancer type}
	\label{tab:samplesize}
	\begin{center}
		\begin{tabular}{cc}
			\textbf{Cancer Type}   & \textbf{Total} \\
			\hline
			\rule{0pt}{2.25ex}BLCA & 261   \\
			CESC                   & 224   \\
			COAD                   & 226   \\
			KIRP                   & 124   \\
			LIHC                   & 264   \\
			READ                   & 73    \\
			SKCM                   & 218   \\
			UCEC                   & 482   \\
		\end{tabular}
	\end{center}
\end{table}

\section{Gene expression analysis}
\label{sec:gene_expression_analysis}

The samples were grouped based on the sample BMI status (obese or non-obese group), and a linear model was fitted to the data based on the sample grouping.
Empirical Bayes statisticsc was used to identify the differentially expressed genes between the two groups, and the top scoring genes were listed.
The error rate of the genes were \gls{fdr} controlled for multiple hypothesis testing.

\section{Pathway enrichment analysis}
\label{sec:pathway_enrichment_analysis}

For a given list of differentially expressed genes identified in a data set, the \texttt{camera} function from the \textit{limma} package was used to identify the pathways that were enriched in those genes.
The \gls{go} database was used to identify the enriched pathways.

\section{\gls{svd}}
\label{sec:singular_value_decomposition}

\gls{svd} was used on a given data set to create the summary metagene scores of a given genetic signature.
% TODO: add more details on the components (s, v, d) created when svd was applied to the data.
Any matrix $X$ can be represented in the form:
\begin{equation}
	\label{eq:svd}
	X = UDV'
\end{equation}

\noindent
where $U$ is the *****, $D$ is the ********, and $V'$ is the **** (eigenvectors and eigenvalues??).

\section{Transformation matrix}
\label{sec:transformation_matrix}

Transformation matrix of a genetic signature was created from the \gls{svd} components.
Rearranging \cref{eq:svd}, you get:

\begin{equation}
	\label{eq:transmat}
	V' = U'D^{-1}X
\end{equation}

\noindent
where $U'D^{-1}$ represents the transformation matrix.
The transformation matrix $U'D^{-1}$ was created from the \gls{svd} components, and was used to transform other data set and obtain the metagene in that data set.

\section{Metagene analysis}
\label{sec:metagene_analysis}

\gls{svd} was applied to the training data set to create the metagene and the transformation matrix in the data set.
The transformation matrix from the training data was used to transform the other data sets and to create the metagene of the genetic signature in those data sets.

Each of these metagenes were plotted in a heatmap to see the association between the metagene and the overall gene expression of the genes that were used to create the metagene.
The metagene was also checked for the association between the sample \gls{bmi} status and \gls{bmi} value, which was plotted in a boxplot and scatter plot, respectively.

\section{Gatza pathway metagene direction}
\label{sec:pathway_metagene_direction}














\section{Obesity metagene prediction}
\label{sec:obesity_metagene_prediction}
















\section{Plot creation}
\label{sec:plot_creation}














\subsection{Heatmaps}
\label{sub:heatmaps}














% Do I need to explain how to make boxplots and scatter plots?
% \subsection{Box plots}
% \label{sub:box_plots}

% \subsection{Scatter plots}
% \label{sub:scatter_plots}






% TODO: re-write everything so that it's in past tense
















